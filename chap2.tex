\section{Die Laplace Gleichung}

Eine der wichtigsten partiellen Differentialgleichungen überhaupt ist die Laplace-Gleichung.

Laplace-Gleichung: $\Delta u = 0, x \in \Omega$

Poisson Gleichung: $-\Delta u = f, x \in \Omega, f \colon \Omega \to \R$
mit $\Omega \subset \R^n$.

$ \Delta u = \sum_{i = 1}^^n \frac{\partial_i^2 u}{2 x_i^2}$ Laplace-Operator.

\subsection{physikalische Interpretation}

Sei $u$ Dichte, z.b. eines Feststoffes, Konzetration einer Lösung und $V \subset \Omega$. Dann
$$
\int_{\partial V} F \cdot \nu = 0, \quad F \text{ Fluss}, \nu \text{ äußere Normale}
$$

Divergenzsatz $\implies \int_V \div F = 0$.

Da $V$ beliebig in $\Omega$ gilt $\div F = 0$.

Annahme: Fluss proportianal $\nabla u$, d.h. $F = -a \nabla u$.

Dann: $\div(-a \nabla u) = - a \div \nabla u = - a \Delta u = 0$.

Interpretation
$$
u \hat = \text{ Temperatur / Konzentration} \quad \text{dann} \quad F = - a \nabla u \text{ Fouriergesetz der Wärmeleitung, Diffusionsgesetz}
$$

Um $\Delta u = 0$ zu lösen, benutze \underline{radiale Symmetrie von $\Delta$}

Also $\Omega = \R^n$, wähle Ansatz

$u(x) = v(|x|) = v(r)$ mit $t = |x|$ ($u$ soll radial symmetrisch, also konstant auf Kreisen sein).

Dann: $\frac{\partial r}{\partial x_i} = \frac{2 x_i}{2 |x|} = \frac{x_i}{r},$

also: 
\begin{align*}
  \frac{\partial u}{\partial x_i} &= v'(r) \frac{\partial r}{\partial x_i} = v'(r) \frac{x_i}{r}, i = 1, \dots, n \\
  \frac{\partial^2 u}{\partial x_i^2} &= v''(r) \frac{x_i^2}{r^2} + v'(r)[\frac{1}{r} - \frac{x_i^2}{r^3}]
\end{align*}

$\implies \Delta u = v''(r) + v'(r) \frac{n - 1}{r} \overset{!}{=} = 0$ (beachte: $v$ ist von einer Veränderlichen!)

Für $v' \neq 0$ gilt $(\log(v'))' = \frac{v''}{v'} = \frac{1 - n}{r}$.

Daher: $v' = \frac{c}{r^{n - 1}}$, denn  $(log(r^{1 - n}))' = (1 - n)(\log r)' = \frac{ 1 - n}{r}$.

Also
$$
v(r) = \begin{cases} c \log(r) + c_2, \quad& n = 2 \\ \frac{c_1}{r^{n - 2}} + c_3, \quad n \geq 3, \quad c_1 = \frac{c}{-n + 2} \end{cases}
$$

Dies motiviert:

\subsection{Definition}

Die Funktion $\phi \colon \R^n \setminus \{0\} \to \R$ definiert durch
$$
\phi(x) = \begin{cases} -\frac{1}{2\pi} \log{|x|},  \quad & n = 2 \\ \frac{1}{n(n-2) \omega_n} \frac{1}{|x|^{n - 2}}, \quad & n\geq 3, \quad \omega_n = \operatorname{Vol}(B_1(0)) \end{cases}
$$

heißt \underline{Fundamentallösung des Laplace-Operators} in $\R^n$. 

Für $f \in C_c^2(\R^n)$ betrachte
$$
u(x) := \int_{\R^n} \phi(x - y)f(y) dy = (\phi \ast f)(x).
$$

Dann gilt

\subsection{Theorem}

Seien $f, u$ wie oben definiert. Dann: 
\begin{enumerate}[(a)]
  \item $u \in C^2(\R^n)$
  \item $-\Delta u = f$
\end{enumerate}

\begin{bem}
  Es gibt also eine \underline{explizite} Lösungsformel
\end{bem}

\begin{proof}
  $n \geq 3$

  (a) $u = \phi \ast f$ wohldefiniert, da $f$ kompakten Träger hat.

  $u \in C^2$, da $f \in C_c^2$ (Ana4, Eigenschaften der Faltung)

  (b) Sei $\varepsilon > 0$. Dann 
  $$
  \Delta u(x) = \underbrace{\int_{B_\varepsilon(0)} \phi(y) \Delta f(x - y) dy}_{=: I_\varepsilon} + \underbrace{\int_{R^n \setminus B_\varepsilon(0)} \phi(y) \Delta f(x - y) dy}_{=: II_\varepsilon}
  $$

  $|I_\varepsilon| \leq \norm{D^2 f}_\infty \int_{B_varepsilon(0)} |\phi(y)| dy \overset{\text{Polarkoord.}}{\leq} c_n \int_0^\varepsilon \frac{r^{n - 1}}{r^{n - 2}} dr = \frac{c_n}{2} r^2 |_0^\varepsilon = \frac{c_n}{2} \varepsilon^2$

  {\tiny{$r^{n-1}$ kommt aus der Funktionaldeterminante, die Integrale über Winkel gehen in $c_n$ ein, $\phi$ ist davon unabhängig, da radial.}}
\end{proof}

\subsection{Lemma (Greensche Formel)}

Seien $u,v \in C^2(\overline \Omega)$.
Dann gilt: ($\nu$ äußere Normale an $\partial \Omega$)
\begin{enumerate}[(i)]
  \item $\int_\Omega \nabla v \cdot \nabla u dx = - \int_\Omega u \Delta v + \int_{\partial\Omega} (\partial_\nu v) \cdot u$
  \item $\int_\Omega (u \cdot \Delta v - v \Delta u) dx = \int_{\partial \Omega} (u \cdot \partial_\nu v - v \cdot \partial_\nu u)$
  \item $\int_\Omega \Delta u = \int_{\partial \Omega} \partial_\nu u$ ($\partial u = \nabla u \cdot \nu$)
\end{enumerate}

\begin{proof}
  Satz von Gauß: $\int_\Omega \div F = \int_{\partial \Omega} F \cdot \nu$

  \begin{enumerate}[(i)]
    \item Setze $F = (\nabla v) u$ im Divergenzsatz
    \item mit (i): $0 = \int\nabla u \cdot \nabla v - \int \nabla v \cdot \nabla u = \cdots$
    \item Divergenz-Satz $\implies \int_\Omega u_{x_i x_i} = \int_{\partial \Omega} u_{x_i} \nu^i; $ Summe liefert (iii)
  \end{enumerate}
\end{proof}

\begin{proof}[Fortsetzung des Beweises von 2.3]
  Also 
  \begin{align*}
    II_\varepsilon 
    &= \int_{\R^n \setminus B_\varepsilon(0)} \phi(y) \Delta f(x - y) dy\\
    &\overset{2.4(i)}{=} \underbrace{-\int_{R^n \setminus B_\varepsilon(0)} \nabla \phi(y) \nabla f(x - y) dy}_{=: I_\varepsilon^a}  +  \underbrace{\int_{\partial B_\varepsilon(0)} \phi(y) \frac{\partial f}{\partial \nu}(x - y) d\sigma(y)}_{\#}
  \end{align*}

  $II_\varepsilon^a \overset{2.4(i)}{=} \underbrace{\int_{\R^n \setminus B_\varepsilon(0)} \Delta \phi(y) f(x - y) dy}_{= 0, \text{ da } \Delta \phi = 0} - \int_{\partial B_\varepsilon(0)} \frac{\partial \phi}{\partial \nu}(y) \cdot f(x - y) d\sigma(y)$

  Nun gilt $\nabla \varphi(y) = - \frac{1}{\omega_n \cdot n} \frac{y}{|y|^n}$, für $y \neq 0$ und $\nu = -\frac{y}{|y|} = - \frac{y}{\varepsilon}$.

  {\tiny{$\partial B_\varepsilon(\omega)$ ist Rand von $\R^n \setminus B_\varepsilon(0)$, deshalb das Minus bei der äußeren Normalen! }}

  Also $\frac{\partial \phi}{\partial \nu}(y) = \nu \cdot \nabla \phi(y) = \frac{1}{n \cdot \omega(n)} \frac{|y|^2}{\varepsilon |y|^n} = \frac{1}{n \omega_n \varepsilon^{n - 1}}$.

  {\tiny{$(\#) \leq \norm{\nabla f}_\infty \int_{\partial B_\varepsilon(0)} |\phi(y) | dy \leq c \varepsilon$}}

  $\implies II_\varepsilon^2 = - \frac{1}{n \omega(n) \varepsilon^{n - 1}} \int_{\partial B_\varepsilon(0)} f(x - y) d \sigma(y) = -\frac{1}{|\partial B(x, \varepsilon)|} f(y) d\sigma(y) \to -f(x)$ (Übung)

  {\tiny{$f(y) = f(y) - f(x) + f(x)$ erweitern, $\int f(x) dy$ im Mittelwert gleich $f(x)$, erses Integral $\to 0$ mit Schrankensatz}}

  $\implies -\Delta u(x) = f(x)$ für $\varepsilon \to 0$.
\end{proof}

Betrachte \underline{Mittelwerteigenschaft} harmonischer Funktionen.

Setze 
\begin{align*} 
  \dashint_{B(x,r)} f(y) dy &:= \frac{1}{|B(x,1)| r^n} \int_{B(x,r)} f(y) dy \quad\text{ Mittel von $f$ über $B(x,r)$} \\
  \dashint_{\partial B(x,r)} f(y) dy &:= \frac{1}{n |B(x,1)|r^{n-1}} \int_{\partial B(x,r)} f(y) d\sigma(y) \quad\text{ Mittel von $f$ über $\partial B(x,r)$}
\end{align*}

\subsection{Satz}

Sei $u \in C^2(\Omega)$ harmonisch, $\Omega \subseteq \R^n$ offen.
Dann:
$$
u(x) = \dashint_{\partial B(x,r)} u d\sigma = \dashint_{B(x,r)} u dy\quad\text{ für alle } x,r \text{ mit } B(x,r) \subseteq \Omega
$$

\begin{bem}
  Außergewöhnliche Eigenschaft, vgl. Taylor.
\end{bem}

\begin{proof}
  Setze
  $ \phi(r) := \dashint_{\partial B(x,r)} u(y) d\sigma(y) \overset{\text{Subst}}{=} \dashint_{\partial B(0,1)} u(x + rz) d\sigma(z)$ 
  
  $\overset{\text{Def}}{=} \dashint_{\partial B(x,r)} \partial_\nu u(y) d\sigma(y)$

  $\overset{\text{Green}}{=} \frac{r}{n} \dashint_{B(x,r)} \Delta u(y) dy \overset{\text{harmon}}{=} 0$ für alle $r$.

  $\implies \phi$ konstant und $\phi(r) = \lim_{t \to 0} \phi(t) = \lim_{t \to 0} \dashint_{\partial B(x,t)} u(y) dy \overset{\text{Lebesgue}}{=} u(x)$

  Weiter:

  $\int_{B(x,r)} u(y) dy \overset{\text{Cavalieri}}{=} \int_0^r \int_{\partial B(x,s)} u d\sigma ds \overset{\text{Def } \phi}{=} \int_0^r \phi(s)n \omega_n s^{n - 1} ds$

  $= u(x) \int_0^r n \omega_n s^{n - 1} ds = \omega_n r^n u(x) \implies \dashint_{B(x,r)} u(y) dy = u(x)$
\end{proof}

\subsection{Satz (Umkehrung der Mittelwerteigenschaft)}

Sei $u \in C^2(\Omega)$ mit
$$
u(x) = \dashint_{\partial B(x,r)} u d\sigma \quad\text{für alle } B(x,r) \subseteq \Omega.
$$

Dann ist $u$ harmonisch.

\begin{proof}
  Angenommen $u$ sei nicht harmonisch, dann $\Delta u \neq 0$ und es existiert $B(x,r) \subseteq \Omega$ mit oBdA $\Delta u > 0$ auf $B(x,r)$.

  Sei $\phi$ wie in 2.5. 
  Dann $$\phi'(r) = \frac{r}{n} \dashint_{B(x,r)} \Delta u(y) dy > 0,$$ aber $$\phi(r) = \dashint_{\partial B(x,r)} u(y) d\sigma(y) = u(x)$$ für alle $r$, also $\phi'(r) = 0$ für alle $r$.
  Widerspruch.
\end{proof}

Als Folgerung erhalten wir das \underline{Maximumsprinzip}.

\subsection{Theorem (Maximumsprinzip)}

Sei $\Omega \subseteq \R^n$ offen, beschränkt und $u \in C^2(\Omega) \cap C(\overline\Omega)$ sei harmonisch in $\Omega$. 
Dann
\begin{enumerate}[(i)]
  \item $\max_{\overline\Omega} u = \max_{\partial \Omega} u$
  \item ist $\Omega$ zusammenhängend, $x_0 \in \Omega$ mit $u(x_0) = \max_{\overline\Omega} u$, dann ist $u$ konstant in $\Omega$.
\end{enumerate}

\begin{proof}
  (ii) Sei $x_0 \in \Omega$ mit $u(x_0) = M := \max_{\overline\Omega} u$.
  Wähle $r > 0$ mit $r < \dist(x_0, \partial \Omega)$.

  Mittelwert eigenschaft liefert:

  $M = u(x_0) = \dashint_{B(x_0, r)} u dy \leq M$.

  Es gilt ``$=$'' $\iff$ $u \equiv M$ in $B(x_0, r)$.

  Daher ist $\{x \in \Omega \colon u(x) = M\}$ offen uns abgeschlossen.

  $\overset{\Omega \text{ zush.}}{\implies} \{x \in \Omega \colon u(x) = M\} = \Omega.$

  (ii) $\implies$ (i) \checkmark
\end{proof}

\subsection{Korollar (Eindeutigkeit des Dirichlet Problems)}

Seien $f \in C(\Omega), g \in C(\partial \Omega), \Omega \subseteq \R^n$ offen, beschränkt.
Dann existiert höchsten eine Lösung $u \in C^2(\omega) \cap C(\overline\Omega)$ des Dirichlet Problems
$$
(\text{DP}) \begin{cases} -\Delta u &= f \text{ in }\Omega \\ u &= g \text{ auf } \partial \Omega \end{cases}
$$

\begin{proof}
  Seiein $u_1, u_2$ Lösungen des (DP).

  Setze $w_1 := u_1 - u_2, w_2 := u_2 - u_1$.

  $\implies -\Delta w_1 = 0 = -\Delta w_2$ in $\Omega, w_{1/2} = 0$ auf $\partial \Omega$

  Maximums-Prinzip $\implies$ $\max_{\overline\Omega} w_1 = \max_{\partial\Omega} w_1 = 0$ und $\min_{\overline\Omega} w_2 = \min_{\partial\Omega} w_2 = 0$ $\implies u_1 = u_2$.
\end{proof}

\subsection{Satz (Glattheit ``harmonischer Funktionen'')}

Für $u \in C(\Omega)$ gelte die Mittelwerteigenschaft. 
Dann $u \in C^\infty(\Omega)$.

\begin{bem}
  Obige Aussage besagt: $\Delta u = 0 \implies u \in C^\infty$.

  Speziell: algebraische Struktur von $\Delta$ impliziert, dass \underline{alle} Ableitungen von $u$ existieren!
\end{bem}

\begin{proof}
  Sei $(\varphi_\varepsilon)_{\varepsilon > 0} $\underline{Mollifier} (vgl Ana4), d.h.
  
  $\varphi \in C^\infty(\R^n), \int_{\R^n} \varphi = 1, \varphi $ radial, $\varphi \geq 0$, $\supp \varphi \subseteq B_1(0)$ dann

  $\varphi_\varepsilon(x) := \frac{1}{\varepsilon^n} \varphi(\frac{x}{\varepsilon})$.

  Setze $\varphi_\varepsilon := \varphi_\varepsilon \ast u$ in $\Omega_\varepsilon = \{x \in \Omega \colon \dist(x, \partial \Omega) > \varepsilon \}$

  {\tiny{Falls $\partial \Omega = \emptyset$, setze $\dist(x, \partial\Omega) = 0$ für alle $x$.}}

  Dann: $u_\varepsilon \in C^\infty(\Omega_\varepsilon)$ (Ana 4!)

  Zeige: $u = u_\varepsilon$ in $\Omega_\varepsilon$. Sei $x \in \Omega_\varepsilon$.

  \begin{align*}
    u_\varepsilon(x) &= \int_\Omega \varphi_\varepsilon(x - y) u(y) dy \\
    &\overset{\text{Def. } \varphi_\varepsilon}{=} \frac{1}{\varepsilon^n} \int_{B(x,\varepsilon)} \varphi\left(\frac{|x - y|}{\varepsilon}\right) u(y) dy \\
    &\overset{\text{Cavalieri}}{=} \frac{1}{\varepsilon^n} \int_0^\varepsilon \int_{\partial B(x,r)} \varphi\left(\frac{r}{\varepsilon}\right) u(y) d\sigma(y) dr \\
    &= \frac{1}{\varepsilon^n} \int_0^\varepsilon \varphi(\frac{r}{\varepsilon}) \int_{\partial B(x,r)} u d\sigma dr \\
    &\overset{\text{MWE}}{=} \frac{1}{\varepsilon^n} u(x) \int_0^\varepsilon \varphi(\frac{r}{\varepsilon}) n \omega_n r^{n - 1} dr \\
    &= u(x) \int_0^\varepsilon \varphi_\varepsilon(r) n \omega_n r^{n - 1} dr \\
    &= u(x) \int_{B(0,\varepsilon)} \varphi_\varepsilon (y) dy
  \end{align*}

  $\implies u_\varepsilon = u$ auf $\Omega_\varepsilon$ und somit $u \in C^\infty(\Omega)$.
\end{proof}

\subsection{Bemerkungen}

Weitere Eigenschaften harmonischer Funktionen (ohne Beweis!)
\begin{enumerate}[a)]
  \item Sei $u$ harmonisch in $\Omega$.
    $$
    |D^\alpha u(x_0)| \leq \frac{C_k}{r^{n + k}} \norm{u}_{L^1(B(x_0,r))} \text{ für alle } B(x_0, r) \subseteq \Omega, \alpha \colon |\alpha| = k
    $$
  \item Verallgemeinerung: Satz von Liouville

    $u \colon \R^n \to \R$ harmonisch und beschränkt, dann $u$ konstant.

    \item Harnacksche Ungleichung:
    Sei $u \geq 0$ harmonisch auf $\Omega$. Dann gilt für jede zusammenhängende, offene Menge $V \subset\subset \Omega$ ( $V \subset \overline V \subset \Omega$ ):
    $$ \sup_V u \leq C \inf_V u, \quad \text{mit } C = C(V) $$ 
\end{enumerate}

Betrachte $\Omega \subseteq \R^n$ offen, beschränkt, $\partial \Omega$ glatt.

Betrachte (DP):
$$
\begin{cases}
  -\Delta u &= f \quad\text{in } \Omega \\
  u &= g \quad\text{auf } \partial\Omega
\end{cases}
$$

Sei $u \in C^2(\Omega) \cap C(\overline \Omega), x \in \Omega, \varepsilon > 0$ mit $B_\varepsilon(x) \subseteq \Omega$.

Setze $V_\varepsilon = \Omega \setminus B_\varepsilon(x)$. Wende Satz von Green (2.4 ii) an auf $V_\varepsilon$ mit $u$ und $\phi(x -y)$

$$
\int_{V_\varepsilon} u(y) \underbrace{\Delta \phi(y - x)}_{= 0 \text{ für } y \neq x} - \phi(y - x)\Delta u(y) dy
\overset{\text{Green}}{=} \int_{\partial V_\varepsilon} u(y) \frac{\partial \phi}{\partial \nu}(y - x) - \phi(y - x) \frac{\partial u}{\partial \nu}(y) d\sigma(y)
$$

Da $\partial V_\varepsilon = \partial \Omega \cup \partial B_\varepsilon(x)$ und
$$
\left| \int_{\partial B_\varepsilon(x)} \phi(y - x) \underbrace{\frac{\partial u}{\partial \nu}(y)}_{|\cdot| \leq C \\\text{ wegen $u$ stetig und $\partial B_\varepsilon$ kompakt}} d\sigma(y) \right|
$$
sowie
$$
\int_{\partial B_\varepsilon(x)} u(y) \underbrace{\frac{\partial \phi}{\partial \nu} (y - x)}_{C_n \frac{1}{\varepsilon^{n - 1}} \\\text{vgl. Ende Bew 2.3}} d\sigma(y)
= C \dashint_{\partial B_\varepsilon(x)} u(y) d\sigma(y) \to u(x) \quad\text{für } \varepsilon \to 0
$$
gilt, folgt:
\begin{displaymath}
  u(x) = -\int_\Omega \phi(y - x) \Delta u(y) dy + \int_{\partial \Omega} \phi(y - x) \underbrace{\frac{\partial u}{\partial \nu}(y)}_{\text{unbekannt}} - u(y) \frac{\partial \phi}{\partial \nu}(y - x) d\sigma(y) \tag{1}
\end{displaymath}

\underline{Idee: } Green: Addiere harmonische Funktion, um unbekannten Term zu kompensieren.

Für $x \in \Omega$ setze $\phi^x := \phi^x(y)$ mit $\begin{cases} \Delta \phi^x &= 0 \text{ in } \Omega \\ \phi^x &= \phi(y - x) \text{ auf } \partial \Omega \end{cases}$

Green mit $\phi^x$ liefert: (durch Ausnutzung von $\Delta \phi^x = 0$):
\begin{displaymath}
  - \int_\Omega \phi^x(y) \Delta u(y) dy = \int_{\partial \Omega} u(y) \frac{\partial \phi^x}{\partial \nu}(y) - \underbrace{\phi^x(y)}_{= \phi(y - x)} \frac{\partial u}{\partial \nu}(y) d\sigma(y) \tag{2}
\end{displaymath}

Addiere (1) und (2): Dann gilt via
$$
G(x,y) := \phi(y - x) - \phi^x(y), \quad x \neq y
$$
folglich: (mit $\phi^x = \phi(y - x)$ auf $\partial\Omega$)
\begin{align*}
	u(x) &= -\int_{\Omega} \underbrace{(\phi(y - x) - \phi^x(y))}_{= G(x,y)} \Delta u(y) dy - \int_{\partial \Omega} u(y) ( \frac{\partial \phi}{\partial \nu}(y - x) - \frac{\partial \phi^x}{\partial \nu}(y)) d\sigma(y) \\
	&= -\int_\Omega G(x,y) \underbrace{\Delta u(y)}_{= -f(y)} dy - \int_{\partial \Omega} \underbrace{u(y)}_{= g(y)} \frac{\partial G}{\partial \nu}(x,y) d\sigma(y)
\end{align*}

Schritt vorwärts, da unbekannter Term verschwunden. Also bewiesen:

\subsection{Theorem (Darstellungsformel von Green)}

Sei $u \in C^2(\overline\Omega)$ eine Lösung von (DP) mit $f \in C(\overline\Omega), g \in C(\partial \Omega)$, dann gilt:
$$
u(x) = \int_\Omega G(x,y) f(y) dy - \int_{\partial \Omega}  \frac{\partial G}{\partial \nu}(x,y) g(y) d\sigma(y).
$$

\subsection{Bemerkungen}

\begin{enumerate}[a)]
  \item erhalten Lösung von (DP), falls $G$ bekannt.
  \item $G \colon \Omega \times \Omega \to \C$ definiert durch 
	$$G(x,y) := \phi(y - x) - \phi^x(y), x \neq y$$
	heißt \underline{Greensche Funktion}
  \item Die Greensche Funktion ist symmetrisch, d.h.
	$$
	G(x,y) = G(y,x) \text{ für } x \neq y
	$$
	{\tiny{Greensche Formeln auf $\Omega \setminus (B_\varepsilon(x) \cup B_\varepsilon(y))$ anwenden und Grenzwert nehmen}}
	Beweis: Übung.
  \item Bestimmung einer Greenschen Funktion ist im Allgemeinen schwierig, betrachte daher Spezialfälle:
    \begin{enumerate}[i)]
      \item Halbraum $\R^n_+ := \{x \in \R^n \colon x_n > 0\}$
      \item $\Omega = B_1(0)$
    \end{enumerate}
\end{enumerate}

\underline{zu i):} Für $x \in \R_+^n$ definiere Reflexion auf $\partial \R_+^n$ via 
$$
\tilde x := (x-1, \dots, x_{n - 1}, -x_n)
$$
Ansatz für Greensche Funktion:
$$
\phi^x(y) := \phi(y - \tilde x).
$$
Dann $\phi^x(y) = \phi(y - x)$ für $y \in \partial R_+^n$, denn
$$
\phi^x(y) = \phi(y_1 - x_1, \dots, \underbrace{y_n}_{=0} + x_n) = \phi(y_1 - x_1, \dots, y_n + x_n),
$$
"da der Betrag in $\phi$ das Minus nicht sieht"
und deshalb
\begin{align*}
	\Delta \phi^x &= 0 \quad\text{in } R_+^n \\
	\phi^x &= \phi(y - x) \quad\text{auf } \partial\R_+^n
\end{align*}
{\tiny{$x \in \R_+^n \implies \tilde x \in \partial \R_+^n \implies y - \tilde x \neq 0$}}

\subsection{Definition}

Die Greensche Funktion für $R_+^n$ ist gegeben durch:
$$
  G(x,y) := \phi(x - y) - \phi(y - \tilde x), \quad x, y \in \R_+^n, x \neq y.
$$
Weiter:
$$
  \frac{\partial G}{\partial \nu}(x,y) \overset{\text{Übung}} = - \frac{1}{n \omega_n} \frac{2 x_n}{|x - y|^n}, \quad x, y \in \R_+^n
$$
und via Theorem 2.11 erwarten wir für $\begin{cases} -\Delta u &= 0 \text{ auf } \R_+^n \\ u&= g \text{ auf } \partial \R_+^n \end{cases}$, dass Lösung $u$ die Gestalt
$$
  u(x) = \frac{2 x_n}{n \omega(n)} \int_{\partial \R_+^n} \frac{g(y)}{x - y}^n dy
$$
hat.

\subsection{Satz}

Sei $g \in C(\R^{n  - 1}) \cap L^\infty(\R^{n - 1}$ und $u$ sei definiert wie oben. Dann:
\begin{enumerate}[i)]
	\item $u \in C^\infty(\R^n_+) \cap L^\infty(\R^n_+)$
	\item $\Delta u = 0$ in $\R^n_+$
	\item $\lim_{x \to x_0, x \in \R^n_+} u(x) = g(x_0), x_0 \in \partial \R^n_+$
\end{enumerate}

Beweis: später via Fourier-Trafo, ohne Rechnen.

\subsection{Satz (Eindeutigkeit des (DP) via Energiemethode)}

Seien $u, \tilde u$ Lösungen des (DP). Setze $w:= u - \tilde u$. Dann
\begin{align*}
	\Delta w &= 0 \quad\text{in } \Omega \\
	w &= 0 \quad\text{auf } \partial \Omega
\end{align*}
somit
$$
0 = -\int_\Omega \underbrace{\Delta w}_{= 0 } \cdot w + \int_{\partial \Omega} (\partial_\nu w) \underbrace{w}_{=0} \overset{\text{Green}}{=} \int_\Omega |\nabla w|^2
$$
$\implies |\nabla w| = 0$ in $\Omega$ $\implies w = u - \tilde u = 0$, da $w = 0$ auf $\partial \Omega$.
