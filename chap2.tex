\section{Die Laplace Gleichung}

Eine der wichtigsten partiellen Differentialgleichungen überhaupt ist die Laplace-Gleichung.

Laplace-Gleichung: $\Delta u = 0, x \in \Omega$

Poisson Gleichung: $-\Delta u = f, x \in \Omega, f \colon \Omega \to \R$
mit $\Omega \subset \R^n$.

$ \Delta u = \sum_{i = 1}^^n \frac{\partial_i^2 u}{2 x_i^2}$ Laplace-Operator.

\subsection{physikalische Interpretation}

Sei $u$ Dichte, z.b. eines Feststoffes, Konzetration einer Lösung und $V \subset \Omega$. Dann

$$
\int_{\partial V} F \cdot \nu = 0, \quad F \text{ Fluss}, \nu \text{ äußere Normale}
$$

Divergenzsatz $\implies \int_V \div F = 0$.

Da $V$ beliebig in $\Omega$ gilt $\div F = 0$.

Annahme: Fluss proportianal $\nabla u$, d.h. $F = -a \nabla u$.

Dann: $\div(-a \nabla u) = - a \div \nabla u = - a \Delta u = 0$.

Interpretation
$$
u \hat = \text{ Temperatur / Konzentration} \quad \text{dann} \quad F = - a \nabla u \text{ Fouriergesetz der Wärmeleitung, Diffusionsgesetz}
$$

Um $\Delta u = 0$ zu lösen, benutze \underline{radiale Symmetrie von $\Delta$}

Also $\Omega = \R^n$, wähle Ansatz

$u(x) = v(|x|) = v(r)$ mit $t = |x|$ ($u$ soll radial symmetrisch, also konstant auf Kreisen sein).

Dann: $\frac{\partial r}{\partial x_i} = \frac{2 x_i}{2 |x|} = \frac{x_i}{r},$

also: 
\begin{align*}
  \frac{\partial u}{\partial x_i} &= v'(r) \frac{\partial r}{\partial x_i} = v'(r) \frac{x_i}{r}, i = 1, \dots, n \\
  \frac{\partial^2 u}{\partial x_i^2} = v''(r) \frac{x_i^2}{r^2} + v'(r)[\frac{1}{r} - \frac{x_i^2}{r^3}]
\end{align*}

$\implies \Delta u = v''(r) + v'(r) \frac{n - 1}{r} \overset{!}{=} = 0$ (beachte: $v$ ist von einer veränderlichen!

Für $v' \neq 0$ gilt $(\log(v'))' = \frac{v''}{v'} = \frac{1 - n}{r}$.

Daher: $v' = \frac{c}{r^{n - 1}}$, denn  $(log(r^{1 - n}))' = (1 - n)(\log r)' = \frac{ 1 - n}{r}$.

Also
$$
v(r) = \begin{cases} c \log(r) + c_2, \quad& n = 2 \\ \frac{c_1}{r^{n - 2}} + c_3, \quad n \geq 3, \quad c_1 = \frac{c}{-n + 2} \end{cases}
$$

Dies motiviert:

\subsection{Definition}

Die Funktion $\phi \colon \R^n \setminus \{0\} \to \R$ definiert durch
$$
\phi(x) = \begin{cases} -\frac{1}{2\pi} \log{|x|},  \quad & n = 2 \\ \frac{1}{n(n-2) \omega_n} \frac{1}{|x|^{n - 2}}, \quad & n\geq 3, \quad \omega_n = \operatorname{Vol}(B_1(0)) \end{cases}
$$

heißt \underline{Fundamentallösung des Laplace-Operators} in $\R^n$. 




