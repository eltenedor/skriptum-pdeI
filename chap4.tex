\subsection{Die Wellengleichung}

Betrachten in diesem Abschnitt die Wellengleichung:
$$
\text{(WGL)} u_{tt} - \Delta u = 0, \quad x \in \Omega, t > 0
$$

Versehen mit Rand- und Anfangsbedingungen.

\begin{ntion}
  $\square u = u_{tt} - \Delta u$ "d'Alembert Operator"
\end{ntion}

\begin{itemize}
  \item Lösungen der Wellengleichung (hjyperbolische Gleichung) zeigen ein sehr verschiedenes Verhalten im Vergleich zu parabolischen Gleichungen.
  \item Lösungen nicht in $C^\infty$.
  \item endliche Ausbreitungsgeschwindigkeit
\end{itemize}

\subsection{Physikalische Interpretation}

$u(x,t)  \hat = $ Auslenkung im Punkt $x$ zur Zeit $t$ einer Stange ($n = 1$), einer Membran ($n = 2$), eines Festkörpers ($n = 3$) für $V \subset \Omega$ glatt.
Dann ist $u_{tt}$ die Beschleunigung und innerhalb von $V$ gilt: 
$$
\frac{d^2}{dt^2} \int u dx = \int u_{tt} dx = \text{Masse} \times \text{Beschleunigung}
$$

Äußere Kraft $-\int_{\partial V} F \cdot \nu d \sigma$, $F \hat =$ Kraft auf $\partial V$, $\nu$ Normale.

Newton: $\int_V u_{tt} dx = -\int_{\partial V} F \cdot \nu d\sigma = - \int_V \div F dx$, d.h. $u_{tt} = -\div F$.

In vielen Fällen $F = F(\nabla u)$.

Linearisierung: $F(\nabla u) \approx -a \dot \nabla u$, also: $u_{tt} = a \cdot \Delta u$.

Betrachten nun Lösungsdarstellungen für $n = 1$, d.h.
\begin{align*}
  u_{tt} - u_{xx} &= 0, \quad x \in \R, t > 0 \\
  u(x,0) &= g(x), \quad x \in \R \\
  u_t(x,0) = 0, \quad x \in \R
\end{align*}

\subsection{D'Alembertsche Formel}

Da
$$
(\ast) \quad \left(\frac{\partial}{\partial t} + \frac{\partial}{\partial x}\right)\left(\frac{\partial}{\partial t} - \frac{\partial}{\partial x}\right) u = 0,
$$
Setze
$$
v(x,t) = \left(\frac{\partial }{\partial t} - \frac{\partial}{\partial x}\right) u,
$$
also folgt aus ($\ast$), dass:
$$
v_t + v_x = 0, \quad x \in \R, t > 0
$$

Dies ist eine Transportgleichung und daher gilt:

$v(x,t) = h(x - t)$ mit $h(x) = v(x,0)$

$\implies u_t(x,t) - u_x(x,t) = h(x - t), x \in \R, t > 0$

\begin{displaymath}
  \implies u(x,t) = u(x, t, 0) + \int_0^t h(x + (t - s) - s) ds = g(x + t) + \frac{1}{2} \int_{x - t}^{x + t} h(y) dy \tag{$\ast\ast$}
\end{displaymath}

Wegen $h(x) = v(x, 0) = u_t(x, 0) - u_x(x, 0) = -g'(x)$ folgt aus ($\ast\ast$): $u(x, t) = \frac{1}{2}(g(x + t) + g(x - t))$ und damit

\subsection{Satz (Lösung der Wellengleichung für $n = 1$}

Sei $g \in C^2(\R)$ und $u$ definiert als $u(x,t) = \frac{1}{2}(g(x + t) + g(x - t))$ (d'Alambertsche Formel). Dann:
\begin{enumerate}[(i)]
  \item $u \in C^2(\R \times [0, \infty)$
  \item $u_{tt} - u_{xx} = 0, x \in \R, t > 0$
  \item $u(x,0) = g(x) , u_t(x, 0) = 0, x \in \R$
\end{enumerate}

\subsection{Bemerkungen}

\begin{enumerate}[a)]
\item Obige Darstellung zeigt $g \in C^k \implies u \in C^k$ aber keine Glättung im Unterschied zu parabolischen Gleichungen.
\item Allgemein: $u(x, t) = \frac{1}{2} (g(x + t) + g(x - t) + \int_{x - t}^{x + t} h(s) ds)$ löst
\end{enumerate}
$$
\begin{cases}
u_{tt} - \Delta u &= 0, \quad t > 0, x \in \R \\
u(x,0) &= g(x), u_t(x,0) = h(x), \quad x \in \R
\end{cases}
$$

\subsection{Bemerkung:}

Für $n = 1$ auch Fourier-Ansatz möglich. Für $n > 1$ schwer!

Übungsaufgabe

Für $n > 1$ mehrere Zugänge möglich. Standard-Methode: sphärische Mittel, technisch aufwändig (siehe Evans)

Hier: Rückführung auf parabolische Gleichung

\subsection{$n$ ungerade (Evans S204, Ex.9)}

Sei $u$ eine Lösung von 
$$
\begin{cases} 
u_{tt} - \Delta u &= 0, \quad t > 0, x \in \R^n \\ 
u(x, 0) &= g(x), \quad x \in \R^n \\ 
u_t(x, 0) &= 0, \quad x \in \R^n 
\end{cases}
$$

Annahme $g \in C_c^\infty(\R^n)$. {\tiny{Später unnötig, aber wichtig für die Herleitung}}

Für $t < 0$ setze $u(x, t) = u(x, -t)$, also:
$$
u_{tt} - \Delta u = 0, \quad x \in \R^n, t \in \R\setminus\{0\}
$$

Setze
$$
(\ast) \quad v(x,t) = \frac{1}{(4\pi t)^{\frac{1}{2}}} \int_\R e^{-\frac{s^2}{4 t}} u(x,s) ds, \quad t > 0, x \in \R^n
$$

Dann
\begin{align*}
(4 \pi t)^{\frac{1}{2}} \Delta v(x,t) 
&= \int_\R e^{-\frac{s^2}{4t}} \Delta u(x,s) ds \\
&= \int_\R e^{-\frac{s^2}{4t}} u_{ss}(x,s) ds \\
&\overset{\text{p.I.}}{=} \int_\R \frac{s}{2t} e^{\frac{-s^2}{4t}} 4s(x,s) ds
\end{align*}

Außerdem: $$v_t(x,t) = \frac{1}{(4\pi t)^{\frac{1}{2}}} \int\left( \frac{s^2}{4t^2} - \frac{1}{2t}\right) e^{-\frac{s^2}{4t}} u(x,s) ds$$

$\implies v$ löst $\begin{cases} v_t - \Delta v &= 0, x \in \R^n, t > 0 \\ v(0) &0 g \end{cases}$

$\overset{g \in C^\infty_c}{\implies} $ v beschränkt und $v = G_t \ast g$ ($\ast\ast$).

Vgl. ($\ast$) und ($\ast\ast$):
\begin{align*}
\frac{1}{(4\pi t)^{\frac{1}{2}}} \int_{\R^n} e^{-\frac{|x - y|^2}{4t}} g(y) dy
&= \frac{1}{(4 \pi t)^{\frac{1}{2}}} \int_\R e^{-\frac{s^2}{4t}} u(x, s) ds \\
&= \frac{2}{(4\pi t)^{\frac{1}{2}}} \int_0^\infty e^{-\frac{s^2}{4t}} u(x,s) ds\\
\intertext{mit $J = \frac{1}{4t}$ gitl also:}
(\ast\ast\ast) \quad &= \frac{1}{2} \left(\frac{J}{\pi}\right)^{\frac{n - 1}{2}} \int_0^\infty e^{-Jr^2} r^{n - 1} G(x,r) dr n \omega_n
\end{align*}

mit $G(x,r) = \int_{\partial B(x,r)} g(s) d\sigma(s) \frac{1}{n \omega_n r^{n - 1}}$

Wir schreiben $n = 2k + 1, k \in \N$, dann
\begin{align*}
J^k \int_0^\infty e^{-J r^2} r^{2k} G(x,r) dr 
&= \frac{(-1)^k}{2^k} \int_0^\infty \left[ (\frac{1}{r}\frac{\partial}{\partial r}|^k e^{-Jr^2} \right] r^{2k} G(x,r) dr \\
&= \frac{1}{2^k} \int_0^\infty r\left[(\frac{1}{r} \frac{\partial}{\partial r}|^r r^{2k - 1} G(x, r) \right] e^{-Jr^2} dr.
\end{align*}

Eindeutigkeit der Laplace-Transformation liefert mit $(\ast\ast\ast)$
$$
u(x, t) = \underbrace{\frac{n \cdot \omega_n}{\pi^k 2{k + 1}}}_{= \frac{n(n - 1)}{n!}} t (\frac{1}{t} (\frac{\partial }{\partial t})^k (t^{2k - 1} g(x,t))
$$

Deshalb gilt
$$
u(x,t) = \frac{n(n -1)}{n!} \frac{\partial}{\partial t} \left(\frac{1}{t} \frac{\partial}{\partial t}\right)^{\frac{ n -3}{2}} t^{n - 2} \dashint_{\partial B(x, t)} g d\sigma,
$$
für $x \in \R^n, t > 0$.

\subsection{Satz (Lösung der Wellengleichung für ungerade Raumdimension)}

Sei $n \geq 3$ ungerade, $g \in C^{\frac{n + 1}{2} + 1}(\R^n)$ und $n $ wie oben. Dann
\begin{enumerate}[(i)]
\item $u \in C^2(\R^n \times [0 , \infty))$ $\to$ nicht so glatt wie der Anfangswert
\item $u_{tt} - \Delta u  = 0, x \in \R^n, t > 0$
\item $\lim_{(x, t) \to (\tilde x, 0)} u(x,t) = g(\tilde x, 0), \tilde x \in \R^n$.
\end{enumerate}

Beweis: nachrechnen!

\subsection{Bemerkung}

Für $n = 3$ heißt die obige darstellung auch Kirchhoffsche Formel und es gilt:

Sei $u_0 \in C^3, u_1 \in C^2$, dann $u(x,t) = \frac{\partial}{\partial t}( \frac{1}{4\pi t} \int_{\partial B(x,t)} g d\sigma) + \frac{1}{4\pi t} \int_{\partial B(x,t)} u_1 d\sigma$ löst $\square u = 0, u(x,0) = u_0(x), u_t(x, 0) = u_1(x)$.

\subsection{Bemerkung}

$u(x,t)$ spürt nur Information en über $u_0, u_1$ auf $\partial B(x,t)$ nicht auf $B(x,t)$.

\subsection{Der Fall $n = 2$, die Absteigemethode}

Idee: Wir führen den Fall $n = 2$ auf $n = 3$ zurück.

Sei $u(x_1, x_2, t)$ Lösung der Wärmeleitungsgleichung für $a = 2$. 
Setze $\overline u(x_1, x_2, x_3, t) = u(x_1, x_2, t)$

Dann gilt:
\begin{align*}
\partial_t^2 \overline u = \Delta \overline u \text{ (in $\R^3$) und } 
\overline u(x_1, x_2, x_3, 0) &= \overline u_0(x_1, x_2, x_3) := u_0(x_1, x_2), \\
\partial_t \overline u (x_1, x_2, x_3, 0) &= \overline u_1(x_1, x_2, x_3) := u_1(x_1, x_2)
\end{align*}

$$\overset{4.8}{\implies} u(x,t) = \overline u(x,t) = \frac{\partial}{\partial t} \left(\frac{1}{4 \pi t} \int_{\partial B^3(\overline x, t)} \overline u_0 d\sigma \right) + \frac{1}{4 \pi t} \int_{\partial B^3(\overline x, t)} \overline u_1 d\sigma$$
{\tiny{$\overline x = (x_1, x_2, x_3)$}}

Schreibe Integral um mittels Parametrisierung der oberen Halbebene

$s^+(\overline x, t) = \{ \overline y \in \partial B^3(\overline , t), y_3 \geq 0\}$

$\phi \colon  B(\overline x, t) \to S^+(\overline x, t), (y_1, y_2) \mapsto \phi(y_1, y_2) = (y_1, y_2, \sqrt{t^2 - |x - y|^2}$

$\implies \int_{\partial B^3(\overline, t)} = \int_{S^+ (\overline x, t)} \overline u_0 + \int_{S^- (\overline x, t)} \overline u_0 = 2 \int_{S^+(\overline x, t)} = 2t \int_{B(x,t)} \frac{\overline u_0(y)}{\sqrt{t^2 - |x - y|^2}} dy$

Deshalb gilt:
$$
u(x,t) = \frac{n \cdot (n - 1)}{n!} \frac{\partial }{\partial t} (\frac{1}{t} \frac{\partial }{\partial t})^{\frac{n - 3}{2}} t^{n - 2} \dashint_{\partial B(x,t)} g d\sigma, x \in \R^n, t > 0.
$$

\subsection{Satz (Lösung der Wellengleicbhung in $\R^2$)}

Sei $n = 2$, $u_0 \in C^3(\R^2), u_1 \in C^2(\R^2)$.
Dann ist $u$ definiert als
$$
u(x,t) 
= \left(\frac{\partial}{\partial t} \frac{1}{2\pi} \int_{B(x,t)} \frac{u_0(y)}{\sqrt{t^2 
- |x - y|^2}} dy\right) + \frac{1}{2\pi} \int_{B(x,t)} \frac{u_1(y)}{\sqrt{t^2 - |x - y|^2} } dy
$$
eine Lösung von
$$
\begin{cases}
\square u(x, t) &= 0, \quad t > 0, x \in \R^2 \\
u(x,0) &= u_0(x), \quad x \in \R^2 \\
u_t(x, 0) &= u_1(x), \quad x \in \R^2
\end{cases}
$$

\underline{Erinnerung:}

\begin{align*}
n = 2 \colon  u(t, x) &= \frac{\partial }{\partial t} \frac{1}{2\pi} \int_{B_t(x} \frac{u_0(y)}{\sqrt{t^2 - |x - y|^2}} dy + \frac{1}{2\pi} \int_{B_t(x)} \frac{u_1(y)}{\sqrt{t^2 - |x - y|^2}} dy \\
n = 3 \colon u(t,x) &= \frac{\partial}{\partial t} \frac{1}{4\pi t}\int_{\partial B_t(x)} u_0 d\sigma + \frac{1}{4\pi t} \int_{\partial B_t(x)} u_1 d\sigma
\end{align*}

\subsection{Bemerkung}

Im Gegensatz zu $n = 3$ (hier Lösung abhängig von $\partial B_t(x)$) ist Lösung für $n = 2$ nur abhängig von $B_t(x)$.

$u_0 = 0, u_1 = $"funktion in $y = x$ konzentriert" $\leadsto$ Distribution

\begin{align*}
n = 3 \colon &u(t,x) \neq 0 \iff y \in \partial B_t(x) \\
n = 2 \colon &u(t,x) \neq 0 \iff y \in B_t(x)
\end{align*}

\subsection{Satz (Eindeutigkleit der Wärmeleitungsgleichung)}

Sei $\Omega \subseteq \R^3$ offen, beschränkt, $\partial \Omega$ glatt. Betrachte
$$
(\ast) 
\begin{cases} 
u_{tt} - \Delta u &= f, \quad x \in \Omega, 0 < t < T, Q_t := \Omega \times (0, T], \Gamma_T :=  \overline Q_T \setminus Q_T \\
u(0,x) &= u_0(x), \quad x \in \partial \Omega \\
u_t(0,x) = u_1(x)
\end{cases}
$$

$\implies$ es existiert höchstens eine Lösung $u \in C(\overline \Omega \times [0, T])$ von ($\ast$).

\begin{proof} (Energiemethode)

Sei $\tilde u$ weitere Lösung, setze $w := u - \tilde u$.
\begin{align*}
w_{tt} - \Delta w &= 0 \quad \text{in } Q_T \\
w &= 0 \quad \text{auf } \Gamma_T \\
w_t &= 0 \quad\text{auf } \Omega \times \{0\}
\end{align*}

Definiere Energie:
$$
E(t) := \frac{1}{2} \int_\Omega w_t^2(t,x) + |\nabla w(t,x)|^2 dx
$$

Dann:
$$
E'(t) = \int_\Omega w_t w_{tt} + \nabla w \cdot \nabla w_t dx \overset{\text{Green}}{=} \int_\Omega w_t\underbrace{(w_{tt} - \Delta w)}{=0} dx + \underbrace{0}_{\text{Randterm}} = 0
$$

$\implies E(t) = E(0) = 0 \implies w_t \equiv 0 \equiv \nabla w$ in $\Omega \times (0, T] \implies w \equiv 0$. 

Definiere:
$$
C_{x_0, t_0} := \{ (x,t) \in \R^n \times \R_+  \colon 0 < t \leq  t_0, |x - x_0 \leq t_0 - t \}
$$
\end{proof}

\subsection{Satz (endliche Ausbreitungsgeschwindigkeit)}

Es gelte $u \equiv u_t \equiv 0$ in $B_{x_0}(t_0) \times \{0 \}$. Dann $u \equiv 0$ in $C_{x_0, t_0}$.

\begin{proof}
$$
E(t) := \frac{1}{2} \int_{B_{x_0}(t_0 - t)} (u_t^2 + |\nabla u|^2) dx
$$

Dann
\begin{align*}
E'(t) 
&\overset{Anhang C.4 Thm 6 Evans}{:=} \int_{B_{x_0}}(t_0 - t) u_t u_{tt} + \nabla u \cdot \nabla u_t dx - \frac{1}{2} \int \int_{\partial B_{x_0}(t_0 - t)} (u_t^2 + |\nabla u|^2) d\sigma \\
&= \int_B u_t \underbrace{(u_{tt} - \Delta u)}_{=0} dx + \int_{\partial B} \frac{\partial u}{\partial \nu} u_t d\sigma - \frac{1}{2} \int_{\partial B} (u_t^2 + |\nabla u|^2 d\sigma \\
&\leq 0
\end{align*}
{\tiny{$|\cdot| \overset{\text{C.S.}}{\leq} C|u_t| |\nabla u|$}}

$\implies E(t) \leq E(0) = 0$ für alle $t \in (0,t_0) \implies u_t \equiv 0 \equiv \nabla u$

$\implies u \equiv 0$ in $C_{x_0, t_0}$.
\end{proof}

