\section{Die Wärmeleitungsgleichung}

Ziel dieses Abschnitts: untersuche die Wärmeleitungsgleichung.
$$
\begin{cases}
  u_t - \Delta &= f, \quad x \in \Omega, t > 0 \\
  u|_{\partial\Omega} &= g, \quad x \in \partial \Omega, t > 0 \\
  h(0,x) &= u_0(x), \quad x \in \Omega
\end{cases}
$$

gegeben: $f, g, u_0$, gesucht: $u \colon [0, \infty) \times \overline \Omega \to \R$

\subsection{Physikalische Interpretation}

$u \hat = $ Konzentration eines bestimmten Stoffes in $\Omega$.

$V \subset \Omega:$

$$
\frac{\partial}{\partial t} \int_V u dx = -\int_{\partial V} \underbrace{F}_{= \text{Flussdichte}} \cdot \nu d\sigma
$$

Satz von Gauß $\implies$ $u_t = -\div F$.

Im einfachsten Fall: $F \sim \nabla F, F = -a\nabla u, a > 0$.

Einsetzen liefert: 
$$
u_t = \div a \nabla u = a \div \nabla u = a \Delta u
$$
d.h.
$$
u_t = \Delta u
$$
ist Modellgleichung.
  
\subsection{Die Fundamentallösung}

Betrachte Wärmeleitungsgleichung auf $\R^n$, d.h.
$$
\begin{cases}
  u_t - \Delta u &= 0, x \in \R^n, t > 0 \\
  u(x,0) &= u_0(x), x \in \R^n
\end{cases}
$$

Wende Fourier-Transformation an auf x.

Sei $u_0 \in \Sc(\R^n)$ (Schwatzraum). Dann gilt:
$$
\begin{cases}
  \hat u_t(t, \xi) + |\xi|^2 \hat u (t, \xi) &= 0, t > 0 \\
  \hat u(0,\xi) &= \hat u_0 (\xi)
\end{cases}
$$

Dies ist gewöhnliche Differentialgleichung, explizit lösbar mit
$$
\hat u (t, \xi) = e^{-t |\xi|^2} \hat u_0(\xi)
$$

Faltungssatz der Fourier-Transformation: ($\hat f \cdot \hat g = (f \ast g)^{\hat{}}$)
$$
  u(t,x) = (G_t \ast u_0)(x)
$$
mit
$$
\hat G_t(\xi) = e^{-t|\xi|^2}
$$
Ana II: 
$$
G(t,x) = \frac{1}{(4\pi t)}^{\frac{n}{2}} e^{-\frac{|\xi|^2}{4t}}
$$

\subsection{Definition}

Die Funktion $G \colon \R \times \R^n \to (0, \infty)$ gegeben durch
$$
G(t,x) = \begin{cases}
  \frac{1}{(4\pi t)}^{\frac{n}{2}} e^{-\frac{|\xi|^2}{4t}}, &\quad t > 0 \\
  0, &\quad t < 0
\end{cases}
$$
heißt \underline{Fundamentallösung} der Wärmeleitungsgleichung oder \underline{Gauß-Kern}

\subsection{Bemerkungen (Eigenschaften von $G$)}

\begin{enumerate}[a)]
  \item $G_t(x) = t^{-\frac{n}{2}} G_1\left(\frac{x}{\sqrt{t}}\right)$
  \item $\int_{\R^n} G_t(x) dx = \hat G_t(0) = 1$
  \item Dies bedeutet, dass $(G_t)_{t > 0}$ ein Mollifier ist.
\end{enumerate}

\subsection{Theorem}

Sei $u$ gegeben durch
$$
u(t,x) = (G_t \ast u_0)(x), x \in \R^n, t> 0,
$$
wobei
\begin{enumerate}[a)]
  \item $u_0 \in \BUC(\R^n)$. Dann gilt:
    \begin{enumerate}[(i)]
      \item $u \in C^\infty(\R^n \times (0,\infty))$
      \item $u_t - \Delta u = 0, x \in \R^n, t> 0$
      \item $\lim_{t \to 0} u(t,x) = u_0(x), x \in \R^n$
    \end{enumerate}
  \item $u_0 \in L^p(\R^n), 1 \leq p < \infty$. Dann
    \begin{enumerate}[(i)]
      \item $u \in C^\infty(\R^n \times (0, \infty))$
      \item Wärmeleitungsgleichung (klassisch) erfüllt.
      \item $\norm{u(\cdot, t) - u_0} \to 0$ für $t \to 0$.
    \end{enumerate}
\end{enumerate}

Beweis: Übungsaufgabe $\leadsto$ Mollifier

\subsection{Bemerkung}

Sei $u_0 \in \BUC(\R^n), u_0 \geq 0, u_0 \neq 0$

$\implies u(t,x) = (G_t \ast u_0)(x)$ ist \underline{strikt} positiv für alle $x \in \R^n, t > 0$!!

``unendliche Ausbreitungsgeschwindigkeit''

Übungsaufgabe.

Betrachte Inhomogenes Anfangsproblem
$$
\begin{cases}
  u_t - \Delta u &= f, \quad x \in \R^n, t > 0 \\
  u(0,x) &= 0, \quad x \in \R^n
\end{cases}
$$
und sei
$$
u(t,x) := \int_0^t \int_{\R^n} G(x - y, t - s)f(y,s) dy ds, \quad t> 0, x \in R^n
$$
mit $f \in C_c^{2,1}(\R^n \times[0,\infty))$

\subsection{Satz (Lösung der inhomogenen Wärmeleitungsgleichung / Prinzip von Duhamel)}

Seien $u, f$ wie oben. Dann:
\begin{enumerate}[(i)]
  \item $u \in C^{2,1}(\R^n \times [0, \infty))$
  \item $u_t - \Delta u = f, x \in \R^n, t> 0$
  \item $u(0,x) = 0, x \in \R^n$
\end{enumerate}

\begin{proof}
  Variablentransformation:
  $u(x,t) = \int_0^t \int_{\R^n} G(y,s) f(x - y, t - s) dy ds$

  $\leadsto u_t(x,t) \overset{\text{Leibniz}}{=} \int_0^t \int_{\R^n} G(y,s) f_t(x - y, t - s) dy ds + \int_{\R^n} G(y,t) f(x - y, 0) dy$

  $\Delta (x,t) = \int_0^t \int_{\R^n} G(y,s) \Delta f(x - y, t - s) dy ds$

  $\implies u_t, D^2 u$ (Ableitung nach $x$) stetig $\implies$ (i)

  (ii) 
  \begin{align*}
    u_t(x,t) - \Delta u(x,t) 
    &= \int_0^t \int_{\R^n} G(y,s) (\frac{\partial}{\partial t} - \Delta) f(x - y, t - s) dy ds + \int_{\R^n} G(y,t) f(x - y, 0) dy \\
    &= \int_\varepsilon^t \int_{\R^n} G(y,s) [ - \frac{\partial}{\partial s} - \Delta_y] f(x - y, t - s) dy ds \big\} \quad\text{(I)} \\
    &+ \int_0^\varepsilon \int_{\R^n} G(y,s) [-\frac{\partial}{\partial s} - \Delta y] f(x - y, t - s) dy ds \big\} \quad\text{(II)} \\
    &+ \int_{\R^n} G(y,t) f((x-y), 0) dy \big\} \quad\text{(III)}
  \end{align*}

  $$
  |II| \leq C \int_0^\varepsilon \int_{\R^n} G*y,s) dy ds = C \cdot \varepsilon
  $$
  \begin{align*}
  I &\overset{\text{p.I.}}{=} \int_\varepsilon^t \int_{\R^n} \left( [ \frac{\partial}{\partial s} - \Delta y] G(y, s)\right) f(x - y, t- s) dy dy \\
                          &+ \int_{\R^n} G(y,\varepsilon) f(x-y, t-\varepsilon) dy - \underbrace{\int_{\R^n} G(y,t) f(x - y, 0) dy}{= \text{III}}
  \end{align*}

  $\implies u_t(x,t) - \Delta u(x,t) = \lim_{\varepsilon \to 0} \int_{\R^n} G(y, \varepsilon) f(x - y, t - \varepsilon) dy \underset{\text{Übung}\leadsto\text{Mollifier}}{=} f(x,t)$
\end{proof}

Betrachte \underline{Mittelwerteigenschaft} für parabolische Gleichungen (ähnlich Wärmeleitungsgleichung)

\subsection{Definition}

Sei $\Omega \subset \R^n$ beschränkt, offen, $T>0$.

Definiere \underline{parabolischen Zylinder} $Q_T$ als
\begin{enumerate}[a)]
  \item $Q_T := \Omega \times (0, T]$
  \item Der \underline{parabolische Rand} $\partial Q_T$ ist definiert durch
    $$
    \Gamma_T := \partial Q_T := \overline Q_T \setminus Q_T
    $$
\end{enumerate}

\underline{Bemerkung}
\begin{enumerate}[a)]
  \item $Q_T$ enthält $\Omega \times \{ t = T\}$
  \item parabolischer entält Boden $\times$ vertikale Seite von $Q_T$, aber \underline{nicht} den ``Deckel''
\end{enumerate}

\subsection{Definition}

Sei $x \in \R^n, t \in \R, r > 0$. Dann heißt
$$
E(x, t, r) := \{ (y, s) \in \R^{n + 1}, s \leq t, \phi(x - y, t- s) \geq \frac{1}{r^n} \}
$$
``heat-ball'' oder \underline{parabolische Umgebung}

\subsection{Satz (Mittelwerteigenschaft für Wärmeleitungsgleichung)}

Sei $u \in C^{(2,1)}(Q_T)$ der Wärmeleitungsgleichung und $f = 0$. 
Dann:
$$
u(x,t) = \frac{1}{4r^n} \iint_{E(x,t,r)} u(y,s) \frac{|x - y|^2}{(t - s)^2} dy ds
$$
für alle $E(x,t,r) \subseteq Q_T$.

Beweis: ohne.

Folgerung aus Satz 3.10 ist:

\subsection{Satz (Starkes Maximumsprinzip für Wärmeleitungsgleichung)}

Sei $u \in C^{(2,1)}(Q_T) \cap C(\overline{Q_T})$ Lösung der Wärmeleitungsgleichung in $Q_T$.
\begin{enumerate}[(i)]
  \item $\max_{\overline Q_T} u = \max_{\Gamma_T} u$
  \item Fall $\Omega$ zusammenhängend und $(x_0, t_0) \in Q_T$ existiert mit
    $$
    u(x_0, t_0) = \max_{\overline Q_T} u,
    $$
    dann $u$ konstant in $\overline Q_{t_0}$
\end{enumerate}

\underline{Bemerkung:} (i) $u$ nimmt Maximum entweder auf $\Omega \times \{0\}$ oder $\partial \Omega \times [0, T]$ an (Übungsaufgabe)

\underline{Beweisskizze:} Sei $(x_0, t_0) \in Q_T$ mit $u(x_0, t_0) = M = \max_{\overline Q_T} u$.

$$
u(x_0, t_0) = \frac{1}{4r^n} \iint_{E(x_0, t_0, r)} u(y, s) \frac{|x_0 - y_0|^2}{(t_0 - s)^2} dy ds \leq M
$$

``$=$'' gilt genau dann, wenn $u \equiv M$ in $E(x_0, t_0, r)$

$\implies u(y, s) = M$ für alle $(y, s) \in E(x_0, t_0, r)$

Sei $(y_0, s_0) \in Q_T$ mit $s_0 < t_0$ und $L = \overline{(x_0, t_0), (y_0, s_0)}$.

Sei $r_0 := \min\{ s > s_0 \colon u(x, t) = M$ für alle $(x, t) \in L, s \leq t \leq t_0\}$

Angenommen $r_0 > s_0$.

$\implies u(z_0, r_0) = M$ für ein $(z_0, r_0) \in L \cap Q_T$

$\implies u \equiv M$ auf $E(z_0, r_0, r), r$ klein

$$
E(z_0, r_0, r) \supset L \cap \{ r_0 - \underbrace{\tau}_{\text{für ein $\tau > 0$}} \leq t \leq r_0 \}
$$
Widerspruch.

\subsection{Korollar (Eindeutigkeit in beschränkten Gebieten)}

Sei $f \in C(\overline Q_T), g \in C(\Gamma_T)$: Dann existiert höchstens eine Lösung $u \in C^{2,1}(Q_T) \cap C(\overline Q_T)$ von 
$$
\begin{cases}
  u_t - \Delta u &= f \quad\text{in } Q_T \\
  u &= g \quad\text{auf } \Gamma_T
\end{cases}
$$

\begin{proof}
  $w:= u - \tilde u, u \tilde u$ seien Lösungen.

  Maximumsprinzip $\implies$ Behauptung \checkmark.
\end{proof}

\subsection{Bemerkung: }

Für unbeschränkte Gebiete ist Korollar 3.12 im Allgemeinen nicht mehr richtig.

Es existiert $\underline{u \not\equiv 0}$ Lösung von
\begin{align*}
  u_t - u_{xx} &= 0 \quad\text{in } \R \times (0, \infty) \\
  u(x, 0) &= 0
\end{align*}

Idee: Für $z \in \C$ setze $\varphi(z) = \begin{cases} e^{-\frac{i}{z^2}} &, z \neq 0 \\  0 &, z = 0 \end{cases}$

Setze $u(x, t) := \begin{cases} \sum_{n = 0}^\infty \frac{d^n}{dt^n} \varphi(t) \frac{x^{2n}}{(2n)!} &, t > 0 \\
  0 &, t = 0 \end{cases}$

Dann ``formal'':
\begin{enumerate}[(i)]
  \item $\lim_{t \to 0} u(x, t) = \sum_{n = 0}^\infty \frac{d^n}{dt^n} \varphi(0) \frac{x^{2n}}{(2n)!} = 0$
  \item 
    \begin{align*} 
      \frac{\partial^2u}{\partial x^2} 
      &= \sum_{n = 0}^\infty \frac{d^n}{dt^{n + 1}} \varphi(t) \frac{x^{2n - 2}}{(2n)!} 2n(2n - 1) \\
      &= \sum_{n = 1}^\infty \frac{d^n}{dt^n} \varphi(t) \frac{x^{2(n - 1)}}{(2(n - 1))!} \\
      &= \sum_{n = 0}^\infty \frac{d^{n+1}}{dt^{n+1}} \varphi(t) \frac{x^{2n}}{(2n)!} = \frac{\partial u}{\partial t}
    \end{align*}
\end{enumerate}

\subsection{Satz}

Sei $u \in C^{2,1}(\R^n \times (0, T]) \cap C(\R^n \times [0, T])$ eine Lösung von 
$$
\begin{cases}
  u_t - \Delta u &= 0, \quad x \in \R^n, t > 0 \\
  u(x, 0) &= g(x), \quad x \in \R^n.
\end{cases}
$$

Falls Konstanten $m, w \geq 0$ mit
$$
u(x, t) \leq M e^{w |x|^2}, x \in \R^n, 0 \leq t \leq T
$$
existieren, dann $\sup_{\R^n \times [0, T]} = \sup_{\R^n} g$

Beweis: Übungsaufgabe.

Betrachte (WLG) auf beschränkten Gebieten und unterscheide Randbedingungen:
\begin{enumerate}[a)]
  \item $u = g$ auf $\partial \Omega, t > 0$ (\underline{Dirichletsche} Randbedingung)
  \item $\partial_\nu u = g$ auf $\partial \Omega, t > 0$ (\underline{Neumannsche} Randbedingung)
  \item $\partial_\nu u + cu = 0$ (\underline{Robin}-Rand)
\end{enumerate}

Betrachte (WLG) mit Dirichlet Randbedingung:
$$
\begin{cases}
  u_t - \Delta u &= 0, \quad x \in \Omega, t > 0 \\
  u &= 0, \quad x \in \partial \Omega, t > 0 \\
  u(x,0) &= u_(0), \quad x \in \Omega
\end{cases}
$$

Ansatz über Separation der Variablen.

Sei $u(x,t) = F(x) G(t)$.
Dann

$0 = u_t - \Delta u = FG' - (\Delta F) G$

$\implies \frac{G'}{G} = \frac{\Delta F}{F}$, wobei die linke Seite nur von $t$ und die rechte Seite nur von $x$ abhängt.

\begin{align*}
  \implies G' &= \lambda G, \quad G(t) = c e^{\lambda t} \\
  \Delta F = \lambda F, \quad F(x) = ?
\end{align*}

\underline{Falls} wir eine Orthonormalbasis $\{F_j\}$ von $L^2(\Omega)$ finden mit ``$\Delta F_j = \lambda F_j$'', $F_j(x) = 0$ auf $\partial \Omega$, so ist $u_0 = \sum \alpha_j F_j$ und $u(x,t) = \sum \alpha_j F_j(x) e^{\lambda_j t}$ Lösungkandidat.

Schwierigkeiten beim Beweis:

Reihen konvergent: $\leadsto \alpha_j, \Omega$

analog: Neumann-Rand: 

Finge ONB von $L^2(\Omega)$ mit $\begin{cases} \Delta F_j &= \lambda_j F_j \\ \partial_\nu F_j = 0 \end{cases}$

Alles heikel ...

\subsection{Bessel- und Riesz Potentiale}

Erinnerung: 
\begin{align*}
  (\Delta t)^{\hat{}}(\xi) &= -|\xi|^2 \hat f(\xi) \\
  (\varphi \ast G_t)^{\hat{}}(\xi) &= e^{-t|\xi|^2} \hat \varphi(\xi)
\end{align*}

Definiere daher: $e^{t\Delta} \varphi := G_t \ast \varphi$

Sei $f \colon [0, \infty] \to \R, |f(t)| \leq M e^{wt}, t \geq 0$
$$
  \tilde f(\lambda) := \int_0^\infty e^{-\lambda t} f(t) dt, \quad \lambda > \omega
$$

Idee: Ersetze $\lambda$ durch $-\Delta$.
$$
\tilde f(-\Delta) = \int_0^\infty e^{\Delta t} f(t) dt
$$
{\tiny{Faltungskern: $\int_0^\infty G(x,t) f(t) dt$}}

\underline{Beispiel 1:} $\lambda^{-\beta} = \int_0^\infty e^{-\lambda t} \frac{t^{\beta - 1}}{\Gamma(\beta)} dt, \beta > 0$

also ist $ (-\Delta)^{-\beta}$ von der Form:
$$
\int_0^\infty G(x, t) \frac{t^{\beta - 1}}{\Gamma(\beta)} dt
$$

Für $0 < \beta < \frac{n}{2}$, so konvergiert obiges Integral
\begin{align*}
  \int_0^\infty G(x,t) \frac{t^{\beta - 1}}{\Gamma(\beta)} dt 
  &= \frac{1}{\Gamma(\beta)(4\pi)^{\frac{n}{2}}} \int_0^\infty e^{-\frac{|x|^2}{4t}} t^{\beta - 1 - \frac{n}{2}} dt \\
  &= \frac{1}{\Gamma(\beta)} \frac{1}{(4\pi)^{\frac{n}{2}}} \int_0^\infty e^{-\sigma}\sigma^{(\frac{n}{2} - \beta - 1)} d\sigma \left(\frac{4}{|x|^2}\right)^{\frac{n}{2}- \beta} \\
  &= c_n \frac{1}{|x|^{n - 2\beta}}
\end{align*}
 mit $\tau = \frac{1}{t}, \sigma = |x|^2 \frac{\tau}{4}$.

Speziell $n > 2, \beta = 1$

$= \frac{1}{(n - 2) \omega_n} \frac{1}{|x|^{n - 2}} \hat = $ Fundamentallösung von $\Delta$.

allgemein $\alpha = 2\beta:$
$$
R_\alpha := c_n \frac{1}{|x|^{n - \alpha}}
$$
heißt \underline{Riesz-Potential} der Ordnung $\alpha$.

\underline{Beispiel 2:} $(\lambda + 1)^{-\beta} = \int_0^\infty e^{-(\lambda + 1)} \frac{t^{\beta - 1}}{\Gamma(\beta)} dt$

$B_\alpha(x) := \frac{1}{\Gamma(\frac{\alpha}{2})} \int_0^\infty G(x,t) e^{-t} t^{\frac{\alpha}{2} - 1} dt$ Besselpotential

und $(\id - \Delta)^{-\frac{\alpha}{2}} f = f \ast B_\alpha$


