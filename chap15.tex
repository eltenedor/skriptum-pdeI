\section{Hopfsches Maximumsprinzip}

Betrachte hier elliptische Operatoren 2. Ordnung, d.h. Operatoren der Form

\begin{enumerate}[1)]
  \item $Au = - \underbrace{\sum_{i,j = 1}^n \partial_j\left(a_{ij}(x) \partial_i u(x)\right)}_{\div((a_{ij})\nabla u} + \underbrace{\sum_{j = 1}^n b_i(x)}_{b \nabla u} \partial_i u(x) + c(x) u(x)$
  \item $Au = - \sum_{i,j = 1}^n a_{ij}(x) \partial_j \partial_i u(x) + \sum_{i = 1}^n b_i(x) \partial_i u(x) + x(x) u(x)$.
\end{enumerate}
mit gegebenen Funktionen $a_{ij}, b_{ij}, c$ auf einem Gebiet $\Omega$.

Operatoren der Form 1) heißen Operatoren in  Divergenzform, währen Operatoren der Form 2) Operatoren in Nicht-Divergenz-Form heißen.

Wir nehmen Symmetrie an, $a_{ij} = a_{ji}$.

\subsection{Definition}

Der Operator $A$ heißt \underline{gleichmäßig elliptisch}, falls ein $\mu > 0 $ existiert mit
$$
\sum_{i,j = 1}^n a_{ij}(x) \xi_i \xi_j \geq \mu |\xi|^2
$$
für alle $x \in \Omega, \xi \in \R^n$.

Anmberkung
\begin{enumerate}[a)]
  \item Obige Definition besagt, dass die symmetrische Matrix $(a_{ij}) =: A$ positiv definit ist mit kleinstem Eigenwert $\geq \mu$.
  \item $a_{ij} = \delta_{ij}, b_i = c = 0$, dann $A = -\Delta$.
  \item Existieren schwache Lösungen von $\begin{cases} Au = f \text{ in } \Omega \\ u = 0 \text{ auf } \partial \Omega\end{cases}$ folgt wie zuvor für $\Delta$.
\end{enumerate}

Weitere Eigenschaften sind schwieriger (höhere Regularität notwendig).

Wenden uns num dem Maximumsprinzip zu:

Betrachte Operatoren in Nicht-Divergenzform mit $c = 0$, d.h. 
$$
Au = - \sum_{i,j = 1}^n a_{ij}(x) \partial_i \partial_j u(x) + \sum_{i = 1}^n b_i(x) \partial_i u(x)
$$
mit 
\begin{itemize}
  \item stetigen Koeffizienten $a_{ij}, b_i$
  \item $A$ glm. elliptisch
  \item $a_{ij} = a_{ji}$
  \item $\Omega \subseteq \R^n$ offen, beschränkt.
\end{itemize}

\subsection{Satz (schwaches Maximumsprinzip)}

Sei $u \in L^2(\Omega) \cap C^0(\overline \Omega)$ so, dass $Au \leq 0$ in $\Omega$. Dann $\max_{x \in \overline\Omega} u(x) = \max_{x \in \partial\Omega} u(x).$

\begin{bem}
  Eine Funktion $u$ mit $Au \leq 0$ in $\Omega$ heißt Unterlösung; analog: $u$ mit $A u \geq 0$ heißt Oberlösung.
\end{bem}

