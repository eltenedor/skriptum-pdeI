\section{Hopfsches Maximumsprinzip}

Betrachte hier elliptische Operatoren 2. Ordnung, d.h. Operatoren der Form

\begin{enumerate}[1)]
  \item $Au = - \underbrace{\sum_{i,j = 1}^n \partial_j\left(a_{ij}(x) \partial_i u(x)\right)}_{\div((a_{ij})\nabla u} + \underbrace{\sum_{j = 1}^n b_i(x)}_{b \nabla u} \partial_i u(x) + c(x) u(x)$
  \item $Au = - \sum_{i,j = 1}^n a_{ij}(x) \partial_j \partial_i u(x) + \sum_{i = 1}^n b_i(x) \partial_i u(x) + x(x) u(x)$.
\end{enumerate}
mit gegebenen Funktionen $a_{ij}, b_{ij}, c$ auf einem Gebiet $\Omega$.

Operatoren der Form 1) heißen Operatoren in  Divergenzform, währen Operatoren der Form 2) Operatoren in Nicht-Divergenz-Form heißen.

Wir nehmen Symmetrie an, $a_{ij} = a_{ji}$.

\subsection{Definition}

Der Operator $A$ heißt \underline{gleichmäßig elliptisch}, falls ein $\mu > 0 $ existiert mit
$$
\sum_{i,j = 1}^n a_{ij}(x) \xi_i \xi_j \geq \mu |\xi|^2
$$
für alle $x \in \Omega, \xi \in \R^n$.

Anmberkung
\begin{enumerate}[a)]
  \item Obige Definition besagt, dass die symmetrische Matrix $(a_{ij}) =: A$ positiv definit ist mit kleinstem Eigenwert $\geq \mu$.
  \item $a_{ij} = \delta_{ij}, b_i = c = 0$, dann $A = -\Delta$.
  \item Existieren schwache Lösungen von $\begin{cases} Au = f \text{ in } \Omega \\ u = 0 \text{ auf } \partial \Omega\end{cases}$ folgt wie zuvor für $\Delta$.
\end{enumerate}

Weitere Eigenschaften sind schwieriger (höhere Regularität notwendig).

Wenden uns num dem Maximumsprinzip zu:

Betrachte Operatoren in Nicht-Divergenzform mit $c = 0$, d.h. 
$$
Au = - \sum_{i,j = 1}^n a_{ij}(x) \partial_i \partial_j u(x) + \sum_{i = 1}^n b_i(x) \partial_i u(x)
$$
mit 
\begin{itemize}
  \item stetigen Koeffizienten $a_{ij}, b_i$
  \item $A$ glm. elliptisch
  \item $a_{ij} = a_{ji}$
  \item $\Omega \subseteq \R^n$ offen, beschränkt.
\end{itemize}

\subsection{Satz (schwaches Maximumsprinzip)}

Sei $u \in L^2(\Omega) \cap C^0(\overline \Omega)$ so, dass $Au \leq 0$ in $\Omega$. Dann $\max_{x \in \overline\Omega} u(x) = \max_{x \in \partial\Omega} u(x).$

\begin{bem}
  Eine Funktion $u$ mit $Au \leq 0$ in $\Omega$ heißt Unterlösung; analog: $u$ mit $A u \geq 0$ heißt Oberlösung.
\end{bem}

\begin{proof}
  1. Fall: Es gelte die strikte Ungleichung $Au < 0$ in $\Omega$.

  Angenommen es existiert $x_0 \in \Omega$ mit $u(x_0) = \max_{x \in \overline\Omega} u(x)$

  $\implies \nabla u(x_0) = 0, (\partial_i \partial_j u)(x_0)$ ist negativ semi-definit.

  Da $A:=(a_{ij}(x_0))$ symmetrisch und positiv definit, existiert orthogonale Matrix $O$ mit $OAO^T = \diag(d_1,\dots,d_n), d_i > 0$.

  Für $g = x_0 + O(x - x_0)$ gilt: $x - x_0 = O^T(y - x_0)$, daher folgt
  \begin{align*}
    \partial_{x_i} u &= \sum_{k = 1}^n \partial_{y_k} uO O_{ik} \\
    \partial_{x_i} \partial_{x_j} u &= \sum_{k,l = 1}^n \partial_{y_k}\partial_{y_l} u O_{i_k} O_{j_l}, \quad\text{also}\\
    \sum_{i,j=1}^n a_{ij}(x_0) \partial_{x_i} \partial_{x_j} u &= \sum_{k,l = 1}^n \sum_{i,j = 1}^n a_{ij}(x_0) \partial_{y_k} \partial_{y_l} u O_{i_k} O_{j_l} \\
    &= \sum_{k = 1}^n d_k \partial_{y_k} \partial_{y_k} u \leq 0 \\
    \implies Au(x_0) \geq 0,
  \end{align*}
  ein Widerspruch.

  Fall 2: Es gelte $A u \leq 0$ in $\Omega$

  Setze $u_\varepsilon(x) := u(x) + \varepsilon e^{\lambda x_1}, x \in \Omega, \varepsilon > 0, \lambda > 0.$

  Damit ist 
  \begin{align*}
  A u_\varepsilon &= Au + \varepsilon A(e^{\lambda x_1}) = A u + \varepsilon e^{\lambda x_1} (-\lambda^2 a_11 + \lambda b_1) \\
  &\leq Au + \varepsilon e^{\lambda x_1} (-\lambda^2 \mu + \lambda \norm{b_1}_\infty ) < 0
  \end{align*}
  für $\lambda$ hinreichend groß.

  $\overset{\text{Fall 1}}{\implies} \max{x \in \overline\Omega} u_\varepsilon(x)  = \max_{x \in \partial\Omega} u_\varepsilon(x)$.

  $\overset{\varepsilon \to 0}{\implies} \max_{x \in \overline\Omega} u(x) = \max{x \in \partial\Omega} u(x)$.
\end{proof}

Verstärken die Aussage jetzt noch dahingehend, dass eine Unterlösung kein Maximum im Inneren annehmen kann, solange sie nicht konstant ist.

\subsection{Lemma (Hopf)}

Sei $u \in C^2(\Omega)  \cap C^1(\overline\Omega)$ so, dass $A u \leq 0$ in $\Omega$ und es existiert $x_0 \in \partial\Omega$ mit $u(x_0) > u(x), x \in \Omega$.
Weiterhin existiert eine offene Kugel $K \subseteq \Omega$ mit $x_0 \in \partial K$ (innere Kugelbedingung).

$\implies \frac{\partial u}{\partial \nu}(x_0) > 0$, wobei $\nu$ die äußere Normale an $K$ in $x_0$ ist.

\begin{proof}
  Setze $v(x) = e^{-\lambda|x|^2} - e^{-\lambda r^2}$, mit $K = B(0,r), \lambda > 0.$
  Dann
  \begin{align*}
    Av &= e^{-\lambda |x|^2}  \sum_{i,j = 1}^n a_{ij} (- u\lambda^2 x_i x_j + 2 \lambda \delta_{ij}) - e^{- \lambda |x|^2} \sum_{i = 1}^n b_i 2 \lambda x_i \\
    &\leq e^{-\lambda|x|^2} (-u \lambda^2 \mu |x|^2 + 2 \lambda \Spur(A) + 2 \lambda |b||x|)
  \end{align*}

  Betrachte nun den Kreisring $R := B(0,r) \setminus B(0,\frac{r}{2})$.

  $\implies Av \leq 0$ in $R$, wenn $\lambda$ genügend groß.

  Aus $u(x_0) > u(x)$ folgt:
  \begin{align*}
    u(x_0) \geq u(x) + \varepsilon v(x), x \in \partial B(0,\frac{v}{2}) \\
    u(x_0) \geq u(x) + \varepsilon \underbrace{v(x)}_{=0}, x \in \partial B(0,r).
  \end{align*}
  für $\varepsilon$ klein genug.

  Damit folgt zum Einen:
  $$
  A(u + \varepsilon v - u(x_0)) \leq 0 \text{ in } R.
  $$

  Zum Anderen ist
  $$
  u + \varepsilon v - u(x_0) \leq 0 \text{ auf } \partial R.
  $$

  $\overset{\text{15.2}}{\implies} u + \varepsilon v - u(x_0) \leq 0$ in $R$.

  Wegen $u(x_0) + \varepsilon v(x_0) - u(x_0) = 0$ folgt
  $$
  \frac{\partial u}{\partial \nu}(x_0) + \frac{\partial v}{\partial \nu}(x_0) \geq 0,
  $$
  also 
  $$
  \frac{\partial u}{\partial \nu}(x_0) \geq -\varepsilon \frac{\partial v}{\partial \nu}(x_0) = \varepsilon \nabla v(x_0) \frac{x_0}{r} = 2\lambda \varepsilon r e^{-\lambda r^2} > 0
  $$
\end{proof}

\subsection{Theorem (starkes Maximumsprinzip)}

Sei $u \in C^2(\Omega) \cap C(\overline\Omega)$.
Fall $A u \leq 0$ in $\Omega$ und $\max_{x \in \overline\Omega} u(x)$ in einem inneren Punkt von $\Omega$ angenommen wird, so ist $u$ konstant in $\Omega$.

Beweis: Übungsaufgabe.
