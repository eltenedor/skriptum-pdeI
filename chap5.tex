\section{Die Schr�dingergleichung}

Betrachte
$$
\begin{cases}
u_t &= i \Delta u, \quad x \in \R^n, t > 0 \text{( oder $t \in \R$)} \\
u(0,x) &= u_0(x)
\end{cases}
$$

Ansatz via Fourier-Transformation bez�glich $x$.

$\implies \hat u_t(\xi) = -i\xi|^2 \hat u(\xi), \hat u(\xi, 0) = \hat u_0 (\xi)$

$\implies \hat u(t, \xi) = e^{-it |\xi|^2} \hat u_0(\xi)$

$\implies u(t, x) = (\F^{-1}(e^{-it|\cdot|^2}) \ast u_0)(x) = \left( \frac{e^{i \frac{|\cdot|^2}{4t}}}{(4\pi i t)^{\frac{n}{2}}} \ast u_0 \right)(x)$

Setze
$$
e^{it\Delta} f:= \frac{e^{-\frac{|\cdot|^2}{4it}}}{(4\pi t)^{\frac{n}{2}}} \ast f
$$

\subsection{Satz}

Sei $u_0 \in L^2(\R^n)$. Dann ist L�sung der Schr�dingergleichung gegeben durch
$$
u(t,x) := (G_{it} \ast u_0)(x) = \left( \frac{e^{-\frac{|\cdot|^2}{4it}}}{(4 \pi it)^{\frac{n}{2}}} \ast u_0 \right)(x)
$$

\subsection{Satz (Eigenschaften)}

F�r $t \in \R$ gilt:
\begin{enumerate}[a)]
\item $\norm{e^{it\Delta} f}_{L^2} = \norm{f}_{L^2}$
\item $e^{i(t + s)\Delta} = e^{it\Delta} e^{i s\Delta}$ und $(e^{it}Delta)^{-1} = e^{-it\Delte} = \left(\e^{it\Delta}\right)^*$
\item $e^{i 0 \Delta} = I$
\end{enumerate}

Beweis: �bungsaufgabe

\subsection{Lemma (Skalierung)}

Ist $u$ eine L�sung der Schr�gdingergleichung, so sind auch 
\begin{enumerate}[(i)] 
\item $u_1(t,x) = e^{i\theta} u(t, x), \theta \in \R$ 
\item $u_2(t,x) = u(t, A,x)$, $A$ orthogonale $n\times n$-Matrix
\item $u_3(t, x) = \lambda^{\frac{n}{2}} u(\lambda^2 t, \lambda x), \lambda \in \R$
\end{enumerate}
ebenfalls L�sungen der Sch�rdingergleichung.

\subsection{Lemma (Abbildungseigenschaften von $e^{it\Delta}$)}

Sei $t \in \R, t \neq 0, \frac{1}{p} + \frac{1}{q} = 1, p \in [1,2]$.

$\implies$ es existiert $C > 0$:
$$
\norm{e^{it\Delta} f}_{L^q} \leq C t^{-\frac{n}{2} (\frac{1}{p} - \frac{1}{q}) \norm{f}_{L^p}.
$$

\begin{proof}
Satz 5.2: $e^{it\Delta}$ Isometrie in $L^2$.

Weiter:
$$
\norm{e^{it\Delta} f}_{L^\infty} \overset{\text{Young}}{\leq} \norm{\frac{e^{i\frac{|\cdot|^2}{4t}}}{(4\pi t)^{\frac{n}{2}}}}_{L^\infty} \norm{f}_1 \leq C t^{-\frac{n}{2}} \norm{f}_{L^1}
$$
also 
\begin{align*}
  e^{it\Delta} = &L^2 \to L^2 \\ &L^1 \to L^\infty
\end{align*}

$\overline{\text{Riesz-thorin}}{\implies} \norm{e^{it\Delta} f}_{L^q} \leq C |t|^{-\frac{n}{2}(\frac{1}{p} - \frac{1}{q}} \norm{f}_{L^p}$

\subsection{Bemerkungen}
\begin{enumerate}[a)]
\item $e^{it\Delta}$ ist f�r $t \neq 0$ \underline{nicht} beschr�nkt von $L^p(\R^n) \to L^p(\R^n)$, falls $p \neq 2$.
\item Schr�dingergleiochung ist Beispiel einer disp. Gleichung.
\end{enumerate}
\end{proof}


