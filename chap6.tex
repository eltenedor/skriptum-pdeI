\chapter*{Schwache Lösungstheorie in Sobolevräumen}
\section{Elliptische Randwertprobleme: Der Fall $n = 1$}

$$
\text{Dirichletproblem}
\begin{cases}
  -u'' = f \text{ auf } [0,1], f \in ([0,1]) \\
  u(0) = u(1) = 0
\end{cases}
$$

klassische Lösung: $u \in C^2([0,1])$, welches (DP) erfüllt.

Zugang in 4 Schritten

\begin{enumerate}[(A)]
  \item Einführung einer schwachen Lösung $\leadsto$ Sobolevraum
  \item Existenz und Eindeutigkeit einer schwachen Lösung
  \item Reularität der schwachen Lösung
  \item Rückkehr zur klassischen Lösung
\end{enumerate}

$I = (a,b) \subseteq \R, -\infty \leq a < b \leq \infty$

Sei $u \in C^1(\overline I), \varphi \in C_c^\infty(I)$

$$ \int_I u' \varphi dx = \underbrace{u \varphi|_a^b}_{= 0 \text{ wegen kompaktem Träger}} - \int_I u \varphi' dx$$

\subsection{Definition}

Wir definieren \underline{Sobolevraum} $H_1(I)$ via
$$
H^1(I) := \left\{ u \in L^2(\Omega) \colon \text{ es existiert } g \in L^2(I) \text{ mit } \int_I u\varphi' dx = - \int_I g\varphi dy \text{ für alle } \varphi \in C_c^\infty(I)\right\}
$$

Für $u \in H^1(I)$ heißt $Du := g$ die \underline{schwache Ableitung} von $u$.

\begin{bem}
Die Funktion $g$ ist eindeutig bestimmt (Fundamentallemma).
\end{bem}

\begin{ex}
  $u(x) = \frac{1}{2}(|x| + x)$
  
  $\implies u \in H^1(I)$ und $Du = H$ mit $H(x) = 
  \begin{cases}
    1,\quad 0 < x < 1 \\
    0,\quad -1 < x < 0
  \end{cases}$
\end{ex}

Versehe $H^1(I)$ mit Skalarprodukt: $$(u,v)_{H^1} := (u,v)_{L^2} + (u', v')_{L^1}$$ und Norm $$\norm{u}_{H^1} := \left(\norm{u}_{L^2}^2 + \norm{u'}_{L^2}^2\right)^\frac{1}{2}$$

\subsection{Lemma}

$H^1(I)$ ist ein Hilbertraum. Übungsaufgabe.

\subsection{Satz}

Sei $u \in H^1(I)$.
Dann existiert $\tilde u \in C(\overline I)$ mit $\tilde u = u$ fast überall auf $I$ und 
$$
\tilde u(x) - \tilde u(y) = \int_y^x u'(s) ds, \quad x,y \in \overline I.
$$

Beweis: Übungsaufgabe

\subsection{Satz}

Sei $-\infty < a < b < \infty$. Dann ist die Einbettung
$$
H^1(a,b) \hookrightarrow C([a,b])
$$
kompakt.

\begin{proof}
  Zu besprechen
\end{proof}

\subsection{Korollar (partielle Integration in $H^1$)}

Seien $u,v, \in H^1(a,b)$.
Dann $u\cdot v \in H^1(a,b)$ und es gilt:
$$ (uv)' = u'v + u v' \quad\text{sowie}\quad \int_y^x u'v = uv|_y^x - \int_y^x u v'$$
für $x,y \in [a,b]$.

\subsection{Satz}

Sei $-\infty < a < b < \infty, u \in L^2(a,b)$. 
Dann
$$
u \in H^1(a,b) \iff \text{ es existiert } C > 0 \text{ mit } |\int_a^b u\varphi'| \leq C \norm{\varphi}_{L^2}$$
für alle $\varphi \in C_c^\infty(a,b)$.

\begin{proof}
  $\Rightarrow$: \checkmark

  $\Leftarrow$: Betrachten Abbildung
  $$
  f \colon C_c^\infty(I) \ni \varphi \mapsto -\int_a^b u \varphi' dx
  $$

  Dann ist $f$ Linearform, definiert auf dichtem Teilraum von $L^2$

  $\implies$ es existiert stetige Fortsetzung auf $L^2(a,b)$.

  $\overset{\text{R.F.}}{\implies}$ es existiert genau ein $g \in L^2(a,b)$ mit $f(\varphi) = (g,\varphi), \varphi \in L^2$.

  Insb. $-\int u \varphi' = \int g \varphi $ für alle $\varphi \in C_c^\infty(I)$.

  $\overset{\text{Def.}}{\implies} u \in H^1(a,b)$
\end{proof}

\subsection{Definition}

Seien $\infty < a < b < \infty$.
Setze
$$
H_0^1(a,b) := \overline{C_c^\infty(a,b)}_{\norm{\cdot}_{H^1(a,b)}}
$$
und versehe $H_0^1(a,b)$ mit der induzierten Topologie.

\begin{bem}
  Dann ist auch $H_0^1(a,b)$ ein Hilbertraum.
\end{bem}

\subsection{Satz}

Sei $u \in H^1(a,b)$ mit $-\infty < a < b < \infty$.
Dann
$$ u \in H_0^1(a,b) \iff u(a) = u(b) = 0.$$

Beweis Übungsaufgabe.

\subsection{Satz (Poincare)}

Seien $-\infty < a < b < \infty$.
Dann existiert $C > 0$ mit $\norm{u}_{L^2(a,b)} \leq C \norm{u'}_{L^2(a,b)}$ für $u \in H_0^1(a,b)$.

\begin{proof}
  Sei $u \in H_0^1(a,b)$, $a < x < b$.

  $u(x) \overset{\text{6.8}}{=} u(x) - u(a) \overset{\text{6.3}} \int_a^x 1\cdot u'(x) ds$

  $|u(x)|^2 \overset{\text{C.S.}}{\leq}\left( \int_a^x 1 ds \right) \left( \int_a^x |u'(s)|^2 ds \right) \leq (b-a) \norm{u'}_2^2$

  $\implies \norm{u}_2^2 \leq (b-a)^2 \norm{u'}_{L^2}^2 \implies \norm{u}_2 \leq (b-a) \norm{u'}_2$
\end{proof}

\subsection{Definition}

Sei $m \geq 2$.
Setze
$H^m(I) := \{ u \in H^{m-1}(I) \colon u' \in H^{m-1}(I)\}$

\begin{bem}
  $u \in H^m(I) \iff$ es gibt $g_1,\dots,g_m \in L^2(I)$ mit
  $$
  \int_I uD^j \varphi = (-1)^j \int_I g_j \varphi, \quad\varphi \in C_c^\infty(I), j = 1,\dots,m
  $$
\end{bem}

\begin{ntion}
  $D^2u := u'' := (u')', D^mu$ analog.
\end{ntion}

\begin{bem}
  Versehen mit Skalarprodukt
  $$
  (u,v)_{H^m} := (u,v)_{L^2} + \sum_{j = 1}^m (D^j u, D^j v)_{L^2}
  $$
  und zugehöriger Norm
  $$
  \norm{u}_{H^m} := \left( \sum_{j \leq m} \norm{D^j u }_{L^2}^2 \right)^{\frac{1}{2}}
  $$
  ist $H^m(I)$ ein Hilbertraum.
\end{bem}

\subsection{Lemma (Fundamentallemma der Variationsrechnung)}

Sei $\Omega \subseteq \R^n$ offen, $f \in L^1_{\loc}(\Omega$.
Falls
$$\int_\Omega f \varphi = 0 \quad \text{für } \varphi \in C_c^\infty(\Omega),$$
dann: $f = 0$ fast überall in $\Omega$.

Beweis findet sich in Alt Funktionalanalysis.

Zurück zum Dirichletproblem

\subsection{Definition}

Eine \underline{schwache Lösung des (DP)} ist eine Funktion $u \in H_0^1(a,b)$ mit
$$
\int u' v' = \int fv, \quad v \in H_0^1(a,b).
$$

\subsection*{Schritt A: klassische Lösung $\implies$ schwache Lösung}

Sei $v \in H_0^1(a,b), f \in L^2(a,b)$. Dann

$\overset{\text{6.5}}{\implies} -\int u'' v = -u' v |_a^b + \int u' v' = \int f v$

\subsection*{Schritt B: Existenz und Eindeutigkeit einer schwachen Lösung}

Z.z.: Für $f \in L^2(a,b)$ existiert genau ein $u \in H_0^1(a,b)$ mit 
\begin{displaymath}
  \int u' v' = \int f v \tag{$\ast$}
\end{displaymath}

\begin{proof}
  Definiere $a(u,v) := \int_I u' v', u,v \in H_0^1(a,b)$.

  Dann ist $a$ stetige und koerzive Bilinearform auf $H_0^1$, denn

  $|a(u,v)|^2 \overset{\text{Hölder}}{\leq} (\int (u')^2) (\int (v')^2) \leq \norm{u}_{H^1}^2 \norm{v}_{H^1}^2 \implies$ $a$ stetig.

  $a$ koerziv, denn

  $a(u,u) = \int_a^b |u'|^2 = \frac{1}{2} \int |u'|^2 + \frac{1}{2} \int |u'|^2$ 
  
  $\overset{\text{Poincare}}{\geq} \frac{1}{2} \int |u'|^2 + \frac{1}{2c} \int_a^b |u|^2 \geq \tilde C \norm{u}_{H^1}^2, u \in H_0^1(I)$.

  Also ist $a$ stetige, koerzive Bilinearform.

  Betrachte rechte Seite von ($\ast$): Linearform $\varphi \colon v \mapsto \int f v$.
  
  Lax-Milgram $\implies$ es existiert genau ein $u \in H_0^1(a,b)$ mit $a(u,v) = \varphi(v)$ für alle $v \in H_0^1(a,b)$.

  D.h.: $\int_a^b u' v' = \int f v, v \in H_0^1(a,b)$, also schwache Lösung des (DP).
\end{proof}

\subsection*{Schritt C: Regularität}

Zeige: $f \in L^1(a,b), u \in H_0^1(a,b)$ schwache Lösung $\implies$ $u \in H^2(a,b)$.

Denn: $\int u' v = \int f v, v \in C_c^\infty(a,b)$.

$\overset{\text{Satz 6.6 + Hölder}}{\implies} u' \in H^1(a,b) \implies u \in H^2(a,b)$

Weiter $f \in L^2(a,b) \cap C[a,b] \implies u \in C^2[a,b]$, denn:

$u' \in H^1 \implies \int_a^b u' v' = u' v |_a^b - \int_a^b u'' v = \int_a^b fv$

$\implies \int_a^b (f + u'') v = 0, v \in C_c^\infty(a,b)$

$\overset{\text{Fundamentallemma}}{\implies} -u'' = f $ fast überall und da $f$ stetig folgt $u \in C^2([a,b])$.

\subsection*{Schritt D: Rückkehr zur klassischen Lösung}

Sei $u \in C^2(\overline I)$ schwache Lösung des (DP) $\implies$ $u$ klassische Lösung von (DP)

\begin{proof}
  Da $u \in H_0^1(a,b)$ gilt nach Satz 6.8: $u(a) = u(b) = 0$ und 

  $\int u' v' = \int fv, v \in C_c^\infty(a,b) \overset{\text{part. Int}}{\implies} \int (-u'' - f) v = 0, v \in C_c^\infty(a,b)$

  $\overset{\text{Fundamentallemma}}{\implies} -u'' - f = 0$ fast überall.

  $u \in C^2[a,b] \implies -u'' = f$
\end{proof}

Zusammenfassend gilt:

\subsection{Theorem}

\begin{enumerate}[a)]
  \item für alle $f \in L^2(a,b)$ existiert genau eine schwache Lösung des (DP)
  \item ist $f$ zusätzlich stetig, so existiert genau eine klassische Lösung des (DP)
\end{enumerate}
