\section{Fouriertransformation auf $\Sc(\R^n)$}

\subsection{Definition}

Der Raum $\Sc(\R^n)$ ist definiert durch
$$
\Sc = \Sc(\R^n) = \left\{f \in C^\infty(\R^n) \colon |f|_{\alpha,\beta} = \sup_{x \in \R^n} |x^\beta D^\alpha f(x)| \leq \infty \text{ für alle } \alpha, \beta \right\}
$$
und heißt \underline{Raum der schnell fallenden Funktionen}.

\begin{ntion}
  $|f|_m := \sup{|\alpha|\leq m, |\beta| \leq m} |f|_{\alpha, \beta}$
\end{ntion}

\subsection{Definition}

Eine Folge $(F_j) \subseteq \Sc$ konvergiert gegen $f \in S, f_j \to f$ in $\Sc$, falls

$|f_n - f|_m \to 0$ für alle $m \in \N$.

\begin{bem}
  \begin{enumerate}[a)]
    \item $\Sc(\R^n)$ ist Frechet-Raum.
    \item $D(\R^n) \subset \Sc(\R^n)$.
    \item $x \mapsto e^{-|x|^2} \in \Sc \setminus D$.
  \end{enumerate}
\end{bem}

\subsection{Definition}

Sei $u \in \Sc$.
Die Fouriertrafo von $u$ ist definiert durch
$$
\hat u(\xi) \F u(\xi) := \int_{\R^n} e^{-i \langle x, \xi \rangle} u(x) dx, \xi \in \R^n
$$

\subsection{Lemma (Eigenschafen)}

\begin{enumerate}[a)]
  \item $\F$ ist lineare, stetige Abbildung von $\Sc$ nach $\Sc$.
  \item $(D^\alpha)^{\hat{}} (\xi) = (i\xi)^\alpha \hat u(\xi), \xi \in \R^n, u \in \Sc$.
  \item $((-ix)^\alpha u)^{\hat{}} (\xi) = D^\alpha \hat u (\xi), u \in \Sc, x, y \in \R^n$.
\end{enumerate}

Beweis: Übungsaufgabe.

\subsection{Beispiel}

Sei $f(x) := e^{-\frac{|x|^2}{2}}, x \in \R^n$.
Dann: 
$$
\hat f(\xi) = (2\pi)^{\frac{n}{2}} e^{-\frac{|\xi|^2}{2}}, \xi \in \R^n
$$

Mit anderen Worten: $(2\pi)^{\frac{n}{2}}$ ist Eigenwert der Fouriertransformation zum Eigenvektor $f$.

Beweis Übungsaufgabe.

\subsection {Lemma}

Seien $f,g \in \Sc(\R^n)$. Wir definieren

$\tau_y f(x) := f(x-y)$

$m_y f(x) := e^{i \langle x, y \rangle}$

$d_a f(x) := f(ax)$

Dann gilt
\begin{enumerate}[i)]
  \item $(\tau_y f)^{\hat{}} (\xi) = (m_{-y} \hat f) (\xi)$
  \item $(m_y f)^{\hat{}} = (\tau_y \hat f)(\xi)$
  \item $(d_a f)^{\hat{}}(\xi) = |a|^{-n} (d_{\frac{1}{a}} \hat f) (\xi)$
  \item $\int \hat f(x) g(x) = \int f(x) \hat g(x)$
\end{enumerate}

Beweis Übungsaufgabe.

\subsection{Definition (inverse Fouriertransformation)}

Für $f \in \Sc(\R^n)$ definieren wir die inverse Fouriertransformation via
$$
(\F^{-1}(f))(x) = \check f(x) := \frac{1}{(2\pi)^n} \int_{\R^n} e^{i \langle x, \xi \rangle} f(\xi) d\xi.
$$

\subsection{Theorem}

Die Fouriertransformation ist ein Isomorphismus von $\Sc$ nach $\Sc$

Mit anderen Worten $(\hat f)^{\check{}} = f$ f+r alle $f \in S(\R^n)$.

\begin{proof}
  $(\hat f)^{\check{}} (x) := \frac{1}{(2\pi)^n} \int e^{i \langle x, \xi \rangle} \hat f(\xi) d\xi \overset{!}{=} f(x)$.

  Für $\varepsilon > 0$ definieren wir:

  $I_\varepsilon(x) := \frac{1}{(2\pi)^n} \int e^{i\langle x, \xi \rangle} e^{-\frac{\varepsilon^2 |\xi|^2}{2}} \hat f(\xi) d\xi
  = \frac{1}{(2\pi)^n} \int g(\xi)  \hat f(\xi) d\xi$

  Mit $g(\xi) = (m_x d_\varepsilon \varphi) (\xi)$ mit $\varphi(\xi) = e^{-\frac{|\xi|^2}{2}}$.

  $\underset{\text{Beispiel 12.5}}{\overset{\text{Lemma}}{\implies}} \hat g(\eta) = \varepsilon^{-n} (2\pi)^{\frac{n}{2}} e^{-\frac{|\eta - x|^2}{2\varepsilon^2}}$

  Also 
  \begin{align*}
    I_\varepsilon(x) 
    &= \frac{1}{(2\pi)^n} \int g(\xi) \hat f(\xi) d\xi = \frac{1}{(2\pi)^n} \int \hat g(\xi) f(\xi) d\xi \\
    &= \varepsilon^{-n} \frac{1}{(2\pi)^n} \int_{\R^n} e^{-\frac{|\xi - x|^2}{2 \varepsilon^2}} f(\xi) d\xi \\
    &= \frac{1}{(2\pi)^{\frac{n}{2}}} (f\ast \varphi_\varepsilon)(x), \quad \varphi(x) = \frac{1}{\varepsilon^n} \varphi(\frac{x}{\varepsilon}), \varphi(x) = e^{-\frac{|x|^2}{2}}
  \end{align*}

  $(\varphi_\varepsilon)$ Mollifier, d.h.  $I_\varepsilon \to f$ in $\L^p(\R^n)$.

  $\implies$ es existiert $(\varepsilon_l) \subset \R_+ \colon I_{\varepsilon_l}(x) \to f(x)$ fast überall.

  $\overset{\text{Lebesgue}}{\implies} I_\varepsilon(x) \to \frac{1}{(2\pi)^n} \int e^{i\langle x, \xi \rangle} \hat f(\xi) d\xi$ $\implies$ Behauptung.
\end{proof}

\subsection{Bemerkung}

Sei $\tilde f$ gegeben durch $\tilde f(x) = f(-x), f\in \Sc(\R^n)$.
Dann:
$$
\hat{\hat{f}} = (2\pi)^n \tilde f.
$$

\subsection{Theorem}

\begin{enumerate}[(i)]
  \item Seien $f, g \in \Sc$, dann $f \ast g \in \Sc$ mit $(f \ast g)^{\hat{}} = \hat f \cdot \hat g$.
  \item $(f \cdot g)^{\hat{}} = \hat f \ast \hat g$
  \item $\int f \overline g dx = (2 \pi)^{-n} \in \hat f \overline{\hat g} d\xi$ (Parseval/Plancherel)
\end{enumerate}

\begin{proof}
  (i) $f\ast g \in \Sc$ (Übungsaufgabe)
\begin{align*}
    (f \ast g)^{\hat{}}(\xi) &= \int_{\R^n} e^{-i \langle x, \xi\rangle} \int_{\R^n} f(x - y) g(y) dy dx \\
    &= \int_{\R^n} \int_{\R^n} e^{-i \langle (x-y), \xi \rangle} f(x-y) dx e^{-i \langle y, \xi\rangle } g(x) dy  \\
    &= \hat f \cdot \hat g (\xi)
\end{align*}

(ii) Aus (i): $(\hat f \ast \hat g)^{\hat{}} = \hat{\hat{f}} \cdot \hat{\hat{g}} \implies \hat f \ast \hat g = (\tilde f \cdot \tilde g)^{\check{}} (2\pi)^{2n} = (2\pi)^{2n} (f \cdot g)^{\hat{}}$

(iii) Sei $h = (2\pi)^{-n} \overline{\hat g} \implies \hat h(\xi) = (2\pi)^{-n} \int e^{-i \langle x, \xi \rangle} \overline{\hat g}(x) dx$

$\implies \overline{\hat h} = g(\xi)$

$\implies \int f \overline g dx = \int f \hat h = \int \hat f \cdot h = (2\pi)^{-n} \int \hat f \overline{\hat g}$
\end{proof}

\subsection{Beispiel: Wärmeleitungsgleichung}

\begin{enumerate}[a)]
  \item Sei $K_t(x) := \frac{1}{(4\pi t)^{\frac{n}{2}}} e^{-\frac{|x|^2}{4t}} \implies \hat K_t(\xi) = e^{-t|\xi|^2}.$

  \item Betrachte
    $$
    \text{(WLG)} 
    \begin{cases}
      u_t(t,x) - \Delta u(t,x) = 0 &\text{für } t > 0, x \in \R^n \\
      u(0,x) = u_0(x) &\text{für } x \in \R^n
    \end{cases}
    $$
\end{enumerate}

Sei $u_0 \in L^2(\R^n), 1 \leq p < \infty$. 
Wir definieren $u(t,x) = K_t \ast u_0$.

Dann gilt
\begin{enumerate}
  \item $(t,x)  \mapsto u(t,x) \in C^\infty(\R^n \times (0,\infty), \C)$
  \item $(\partial_t - \Delta) u = 0$
  \item $u(t, \cdot) \overset{t \to 0}{\to} u_0$ in $L^p$.
  \item Definiere für $t > 0$: $T(t) \colon L^p \to L^p$ durch $(T(t)u_0)(x) = u(t,x)$.
    Dann löst $T(\cdot)u_0$ (WLG).
  \item $K_s \ast K_t = K_{s + t}$ für alle $s,t > 0$.
  \item $T(t)T(s) = T(t + s)$ für alle $s, t >0$ (Halbgruppeneigenschaft).
  \item $u_0 \in BUC(\R^n) \implies u \in BUC([0,\infty) \times \R^n)$ und $u(0,x) = u_0(x)$.
\end{enumerate}

\subsection{Satz}

Die inverse Fouriertrafo der Funktion$\xi \to e^{-t|\xi|} ( t > 0, \xi \in \R^n)$ ist gegeben durch:
$$
P_t(x) = \frac{\Gamma(n + \frac{1}{2})}{\pi^{\frac{n+1}{2}} \cdot \frac{t}{(t^2 + |x|^2)^{\frac{n + 1}{2}},\quad t > 0, x \in \R^n
$$

\begin{proof}
  1. Schritt: $e^{-\beta} = \int_0^\infty \frac{e^{-s}}{\sqrt{\pi s }} e^{- \frac{\beta^2}{4s}} ds (\beta > 0)$

  2. Schritt: 
  \begin{align*}
    (e^{-t|\xi|}^{\check{}}(x) 
    &= (2\pi)^{-n} \int_{\R^n} e^{i \langle \xi, x \rangle} e^{-t|\xi|} d \xi \\
    &= (2 \pi)^{-n} \int_{\R^n} e^{i \langle \xi, x\rangle} \int_0^\infty \frac{e^{-s}}{\sqrt{\pi s}} e^{-\frac{t^2|\xi|^2}{4s}} ds d\xi \\
    &= \int_0^\infty \frac{e^{-s}}{\sqrt{4\pi s}}(2\pi)^{-n} \int_{\R^n} e^{i \langle x, \xi \rangle} e^{- \frac{|\xi|^2 t^2}{4s}} d\xi ds\\
    &= \int_0^\infty \frac{e^{-s}{\pi s} \frac{s^{\frac{n}{2}}}{(\pi t)^{\frac{n}{2}} e^{-\frac{|x|^2 s}{t^2}} ds \\
    &= \frac{1}{\pi^{\frac{n+1}{2}} \frac{1}{t^{\frac{n}{2}}} \int_0^\infty s^{\frac{n-1}{2}} e^{-s(1 + \frac{|x|^2}{t^2})} ds \\
    &= P_t(x),  \quad \text{ mit } \int_0^\infty e^{-\lambda t} t^\alpha dt = \frac{\Gamma(\alpha + 1)}{\lambda^{\alpha + 1}}
  \end{align*}
\end{proof}

\subsection{Beispiel (Dirichlet-Problem im Halbraum)}

Wir setzen $\R_+^{n+1} := \{ (x,t) \in \R^n \times \R, t > 0\}$.

Dirichlet-Problem: Sei $f \in \Sc(\R^n)$. Finde $u$ mit 
$$\begin{cases} (\Delta_x + \partial_t^2)u = 0, (x,t) \in \R^{n+1} \\ u(x,0) = f(x) \end{cases}.$$

Fouriertrafo bezgl. x liefert: 
